\title{Wave functions in phase space}

\begin{document}

Most potentials have characteristic length scales

Blurs around classical trajectories, with quantised phase stripes.  (How does the phase correspond to action-angle coordinates?)

Immediately obvious why these are stationary states, and why they are quantised.

More physical insight: Airy function shows incoming and outgoing particles transparently.

``Quantisation is always due to some topological invariant''.  Analytic structure of Bargmann function vs quantum number and hamiltonian.  How many degenerate zeros are there?  Are there ever poles?

Numerical analysis.  How to expand a wave function over Lagrange(?) polynomials.

Can we construct approximate eigenstates by sweeping a coherent state around the classical trajectory, and adjusting the size of the trajectory to satisfy some quantum condition?  Can we then construct a better approximation to |13〉 by exact diagonalisation of H over |10〉⋯|15〉?

Is it possible to construct 〈α|φ_i〉 on a classical trajectory with phase e^{iS} and magnitude by continuity, then extend it Bargmann-analytically to get a wave function?  Is this any different from the fluid dynamics limit of wave mechanics, other than making it obvious why things are quantised?

Degeneracy.  In a double well system, you can construct eigenstates for each well separately, then diagonalise the ones in a certain energy range to account for tunnelling.  There is a sense here of "real degeneracy", where the wave functions have to overlap in phase space, e.g. the cw and ccw travelling waves in a ring, and "energy coincidences", such as exactly symmetric double wells where the bound states of each well happen to have the same energy, but that would change under a small pertubation.

\end{document}