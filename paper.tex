\nonstopmode
\input a4
\input xpmath
\input respnotes \tenrm

\title{Stability of variational quantum dynamics with Gaussian approximants}

\def\G{|Γ〉}
\def\A{|A〉}
\def\Gx{|Γ_χ〉}
\def\cite#1{$[\hbox{\tt #1}]$}
\def\T{^{\rm T}}

%% Background

The Gaussian wave packet is a ubiquitous idea in quantum mechanics.  Its discovery was inevitable: people were provoked by understanding that observables are essentially uncertain to study physical states in which they are normally distributed.  The Gaussian wavepacket also crops up as the quantum state which, placed in a quadratic potential, best approximates the harmonic oscillation of classical mechanics.  In this potential, there is a natural width for the packet, which ensures its shape remains constant during its motion.  This width can also be set by requiring equipartition of energy and the uncertainty principle.  Wavepackets with this natural width are coherent states.

It has also been a productive tool.  The simplest mechanical systems allow Schrödinger's law of motion to be solved in closed form, but most quantum problems must be solved approximately.  This can be done variationally, specifying a class of approximants that can be manipulated exactly, then finding one of them that most nearly solves the problem.  If the problem is to find a wave function $|ψ(t)〉$ that evolves according to a Hamiltonian $H$, the quality of approximation can be measured by comparing the left and right sides of Schrödinger's equation, $i\hbar{d|ψ〉\over dt}$ and $H|ψ〉$.  These are both vectors in Hilbert space, and can be compared by the norm of their difference.  The gaussian wave packets have often been used as variational approximants, because they are defined by their width and their central position and momentum, useful properties to determine for any quantum state.

The invention of the laser in 1963 called for a quantum theory of intense radiation.  Laser beams are hard to analyse because their quantum descriptions are so large, comprising up to $10^{18}$ photons distributed across thousands of modes.  The theory was made managable by Glauber and Sudarshan, who showed that any quantum state can be expanded as a superposition of coherent states.  Laser light is very coherent, so these expansions often represent it efficiently.  Moyal(?) developed an efficient way to calculate dynamics, leading to a family of phase space Monte-Carlo methods.  When very coherent matter, in the form of atomic Bose-Einstein condensates, became easy to access in the lab, phase space methods were sucessfully applied to calculate its dynamics.  They have since been applied to other forms of condensed matter.  They have solved many problems, but are limited by intractable convergence issues.

In the 1970s, Heller put variational mechanics and phase space expansions together.  He proposed to simulate quantum dynamics variationally, using finite superpositions of coherent states as the approximants.  This is an appealing approach.  Every quantum state can be approximated arbitrarily well as the number of components in the superposition increases, and the finite superposition converges to a continuous phase space expansion.  The experience of quantum optics shows that, in many systems, the expansion converges when the superposition has a reasonable number of terms.  Also, variational dynamics should avoid the covergence problems that limit the Monte-Carlo methods.

%% Problem

There have been several attempts to implement Heller's proposal, but none has entirely succeeded.  In all cases, numerical instability has prevented the superposition being expanded to the point of convergence.
%% Consequence
These attempts have aimed to solve chemistry problems, involving dozens of atoms.  Their failure to converge limited their precision, but useful results were obtained from superpositions with a few components, and ad-hoc methods to improve their stability.  The problems that people attempt with phase space methods come from atom optics or condensed matter, and involve thousands or millions of atoms.  In these systems, the instability would prevent any useful information being obtained.  Therefore a systematic approach to regularisation is required in order to approach these problems variationally.  The prospective rewards are significant: many-particle problems include high temperature superconductivity and the dynamics of the early universe.

%% Overview

Both condition and numerical stability refer to reasons that a computer can give the wrong answer to a numerical problem, and the distinction between them is a bit subtle.  A problem is ill-conditioned if the solution is very sensitive to the data with which the problem is stated: the canonical example is computing $\sin(x)$ for $x=10⁶$, whence a relative error of $π×10^{-6}$ in $x$ causes a 100\% error in $\sin(x)$.  The key thing to note is that this has nothing to do with how $\sin(x)$ is computed—the error in $x$ could be due to measurement uncertainty, in which case there would be no way to calculate $\sin(x)$: we'd need to measure it directly.  Condition is a property of problems.  Stability is a property of solutions.  For example, if we asked a computer to calculate $1-\cos(x)$ in the obvious way, with $x$ between 0 and $10^{-10}$, the result might vary randomly around 0, in a range determined by the precision with which the computer stored numbers.  But this problem is not intrinsically hard: at any reasonable level of precision, it amounts to finding $x²/2$.  The method of subtracting a numerical value of $\cos(x)$ from 1, which generates random numbers, is unstable.  Experiments with many-particle systems suggest that phenomena such as interference and superconductivity are robust, and changing the system slightly does not greatly change the phenomena that result. So computing the properties of these systems should be a well-conditioned problem.  However, methods of solving them variationally, using coherent states, have been very unstable.

The instability appears to have two major causes.  The first problem occurs when quantum states are expanded over a finite set of coherent states.  There is a natural way to expand a state as an integral over the phase space of coherent states, with a weight function that varies smoothly with the coherent amplitude.  When this function is sampled at a discrete set of coherent amplitudes and the integral approximated by a sum, the state is approximated to some precision.  Part of the error in this appoximation is due to the finite set of coherent states having bounded amplitude, wheras the state being approximated might have components with arbitrarily large particle number.  When expansion weights for the finite set of coherent states are set by least squares, components with large particle number are approximated by adjacent amplitudes having large weights, with opposite signs so that components with low particle numbers cancel.  This causes the expansion weights to be large, to not resemble samples of any smooth function, and to change sharply as the state changes.  All these properties make it difficult to numerically integrate a rate of change that has been expanded by least squares.  The problems are exacerbated when the amplitudes are spaced irregularly, instead of on a grid.

Expanding a field over Gaussian functions is a rather obvious idea, and physicists are not the only people to have thought of it.  The mathematicians and signal theorists who developed wavelet expansions have also analysed Gaussians.  They call this type of expansion a Gabor frame, or a windowed Fourier transform.  They have proved several results that are not well known in physics: for example, that discrete sets of coherent states are complete, not just the continuum.  They have also studied overcomplete sets of vectors in general, under the field of frame theory.  Many of these general ideas insight into the specific case of coherent state frames.

The behaviour of the discrete expansion weights is similar to that observed in discrete ill-posed problems, and the methods used to analyse and resolve these can be applied to expansions over coherent states.  However, the cause of ill-posedness in our problem differs from those usually studied—the large number components of the state are real, while the high-frequency components that play the same role in ill-posed problems are spurious ones introduced by errors in measured data.  Therefore the standard methods of resolution don't work as well as usual.  The expansion problem is easier in one way: it is iterated at each timestep, so the dynamics of the coherent amplitudes can be modified so that the expansion problem becomes better posed as the simulation progresses.

The second type of instability is a dynamical one.  This occurs whenever Schrödinger's equation is integrated with a quartic hamiltonian, over discrete timesteps.  In the second quantised picture, it occurs because the repulsion between particles causes components where many particles occupy the mode to oscillate very rapidly.  The simulation becomes stable when these components are removed, but that changes the dynamics.  Integrating Schrödinger's equation with repulsive particles either requires very short timesteps, or special integration formulae that exploit features of Hamiltonian dynamics.  Applying these formulae to the variational dynamics of the coherent states, which are not Hamiltonian, is somewhat subtle.


\beginsection{Notational preliminaries}

Quantum mechanics deals with infinite dimensional linear algebra, represented abstractly as bras and kets.  Numerical analysis is also founded on vectors and operators, but in finite dimensional spaces, expanded as matrices and column vectors.  The numerical quantum mechanics discussed in this paper is most conveniently described by combining these notations.

Consider a linear combination of kets,
$$|ψ〉=c₁|a₁〉+c₂|a₂〉+⋯+c_R|a_R〉.$$
This can be written as a matrix product $|ψ〉=|A〉c$, where $c=\pmatrix{c₁&⋯&c_R}\T$ is a column vector of coefficients, and $|A〉=\pmatrix{|a₁〉&⋯&|a_R〉}$ is a row vector of kets.  Alternatively, the kets $|a_r〉$ can be regarded as infinite dimensional columns, and $|A〉$ as a matrix, representing a linear operator from ${\bf C}^R$ into ket space.  This operator has an adjoint, $〈A|=\pmatrix{〈a₁|&⋯&〈a_R|}\T $, that maps kets back to column vectors.  Under the rules of matrix multiplication, the product
$$〈A|A〉=\pmatrix{〈a₁|a₁〉&&〈a₁|a_R〉\cr &\ddots&\cr 〈a_R|a₁〉&&〈a_R|a_R〉}$$
operates on ${\bf C}^R$, and $|A〉〈A|=|a₁〉〈a₁|+⋯+|a_R〉〈a_R|$ operates on Hilbert space.  Other matrix products work too; provided the $|a_r〉$ are linearly independent, $|A〉$ will have a singular value decomposition $|A〉=|U〉SV†$, where $|U〉$ is a row of $R$ kets, $S$ is an $R×R$ diagonal matrix of complex numbers, and $V$ is an orthogonal $R×R$ complex matrix%
\footnote{${}^1$}{In fact, $|A〉$ has an SVD regardless, but $|U〉$ and $S$ are smaller.  When the $|a_r〉$ are linearly dependent, they span a space of dimension less than $R$, and there is no way to choose in which directions it should be extended to give $R$ orthonormal kets.  In this case, $|U〉$ will be a row of ${\rm rank}|A〉$ kets, and $S$ will be ${\rm rank}|A〉×R$.}.
In this paper, things written as kets are linear mappings from other things to quantum states.  The other things might be finite vectors of complex numbers, in which case the ket can be thought of as a matrix, or they might be functions from an infinite dimensional space, in which case this is less useful.  Bras are the other way round: they map states to things other than states.  Ordinary bras and kets are a special case.

Even more general matrices are sometimes useful.  For instance, the operator
$$M=\pmatrix{0&|A〉\cr 〈A|&0}$$
acts on columns of the form $\pmatrix{|ψ〉&z₁&⋯&z_R}\T $ to give columns of the same form.  The top left zero is an operator on Hilbert space, while the bottom right one is an $R×R$ matrix.  The general rule is that the things in each row must have the same column dimension, and visa-versa.  Things that have finite dimensional rows and columns will be written as matrices, and things that are infinite in both directions as Hilbert space operators.  These are both just written as letters.

So far, the operator $|A〉$ has been a linear function from vectors into ket space.  General functions will be written $|f〉$ as well, but their applications will be written $|f(c)〉$ instead of $|A〉c$.  If this function is analytic, it has partial derivatives $|∂_if(c)〉$ with respect to the components of $c$.  The derivative $∂_i$ acts on the function, not in Hilbert space, so it is written inside the ket.  These can be assembled into a Jacobian matrix $|Df(c)〉=\pmatrix{|∂₁f(t)〉&⋯&|∂_Rf(t)〉}$, such that $|f(c+h)〉=|f(c)〉+|Df(c)〉h+o(h²)$.  The derivatives $|∂_if〉$ are ket-valued functions like $|f〉$.


\beginsection{Variational dynamics}

In the notation defined above, Schrödinger's equation is
$$i\hbar |DΨ(t)〉=H|Ψ(t)〉.$$
Here, $|Ψ〉$ denotes a ket-valued function of time, and $|Ψ(t)〉$ is the ket that this function evaluates to at time $t$.  The Hamiltonian $H$, a Hilbert space operator, acts on this ket, not on the function $|Ψ〉$, and is written outside the ket.  Schrödinger's equation could be abbreviated to
$$i\hbar |DΨ〉=H|Ψ〉,$$
equating functions instead of their values.

Schrödinger's equation is very simple.  The main challenge of solving it numerically lies in representing the ket $|Ψ(t)〉$—as is well known, the storage required to expand this over a regular basis grows exponentially with the number of particles being simulated, and exceeds the number of atoms in the known universe when the system includes mere thousands of particles.  In order to compute dynamics by propagating a quantum state, we must restrict the form of $|Ψ〉$, and require that
$$|Ψ(t)〉=|ψ(z(t))〉,$$
or $|Ψ〉=|ψ\circ z〉$, where $z(t)=\pmatrix{z₁(t)&…&z_R(t)}\T $ is a vector of complex parameters.  This poses two challenges: to find a form $|ψ〉$ that can approximate the states $|Ψ(t)〉$ with a reasonable number of parameters $R$, and to stably and efficiently compute $z(t)$ to accomplish that approximation.

Some methods of computing dynamics, such as those involving single-particle Green's functions, derive the expectation values of certain observables without storing data from which the quantum state of the system could be derived.  The above model does not apply to these.  There are also methods, such as phase-space Monte-Carlo ones, that have the above form in principle, but which compute different parameters $z_r(t)$ independently, and find expectation values of observables without having to store them all at once.

The next sections will discuss the approximant $|ψ(z)〉$ analysed in this paper.  Before that, I'll outline the method that will be used to compute the dynamical $z(t)$.  The final part of the paper discusses the many details of making this computation numerically stable.

If the approximant $|Ψ(t)〉=|ψ(z(t))〉$ is substituted in Schrödinger's equation, and the chain rule applied, the result is
$$i\hbar|Dψ(z)〉Dz=H|ψ(z)〉.$$
On the left hand side, the kets in the row vector $|Dψ(z(t))〉$ are combined with coefficients from the column vector $Dz(t)=\pmatrix{Dz₁(t)&…&Dz_R(t)}\T $.  This equation is unlikely to have an exact solution: the kets $|∂_rψ(z(t))〉$ span an $R$-dimensional subspace of Hilbert space, and the ket $H|ψ(z(t))〉$ is likely to lie outside this subspace.  In other words, the restricted form of the approximation does not include the exact solution to Schrödinger's equation.  We can, however, look for a least-squares solution, a vector $\dot z(z)$ that minimises the residual
$$\bigl\|i\hbar|Dψ(z)〉\dot z(z)-H|ψ(z)〉\bigr\|²,\tag\residual$$
and solve the ordinary differential equation
$$Dz(t)=\dot z(z(t)),\tag\ode$$
giving a state $|ψ(z(t))〉$ that hopefully approximates the solution to Schrödinger's equation.  When the norm used is the one derived from the inner product of Hilbert space, this is called the Dirac-Frenkel method.

The Dirac-Frenkel solution is only optimised locally.  There might be some $w(t)$ such that, at large times, $|ψ(w(t))〉$ approximates $|Ψ(t)〉$ better that the Dirac-Frenkel solution $|ψ(w(t))〉$ does.  This would require that, at early times, $i\hbar|Dψ(z(t))〉$ is closer to $H|ψ(z(t))〉$ than $i\hbar|Dψ(w(t))〉$ is to $H|ψ(w(t))〉$, but that $z(t)$ loses the ability to track the dynamics at later times.  When we discuss regularisation below, we will see that this is in fact the usual case.  Global optimisation of $z$ is not practical: even computing the local approximation stretches computers to their limits.

The kind of ansatz considered in this paper has the form 
$$|ψ(z₁,…,z_R)〉=|ψ'(z₁)〉+⋯+|ψ'(z_R)〉.$$
Here, $z_r$ may be a vector of complex numbers, perhaps a weight and the amplitudes of several modes.  In fact, this form is very general: the DMRG approximants are the only ones I know of that are not a sum of similar terms.  In this case, the derivative has the form
$$|Dψ(z₁,…,z_R)〉=\pmatrix{|Dψ'(z₁)〉&…&|Dψ'(z_R)〉}.$$

I will refer to the kets $|ψ'(z_r)〉$ as the components of $|ψ(z)〉$.  It is interesting to consider the case where these lie in orthogonal subspaces.  This will very nearly be the case, for instance, if the components are coherent states, with sparsely distributed amplitudes.  In this case, the component derivatives $|Dψ'(z_r)〉$ also lie in the orthogonal subspaces, and the total derivative has a block diagonal structure,
$$|Dψ(z₁,…,z_R)〉=\pmatrix{|Dψ'(z₁)〉&&\cr&\ddots&\cr&&|Dψ'(z_R)〉},$$
where the empty elements in row $r$ are null kets in the subspace of the component $|ψ'(z_r)〉$.  The ket from Equation~\residual\ thus becomes
$$i\hbar|Dψ(z)〉\dot z(z)-H|ψ(z)〉=
	\pmatrix{|Dψ'(z₁)〉{\dot z}₁(z)-H|ψ'(z₁)〉\cr\vdots\cr |Dψ'(z_R)〉{\dot z}_R(z)-H|ψ'(z_R)〉},$$
where the rows are mutually orthogonal.  Therefore, the least squares problem can be solved separately for each component, independently minimising 
$$\bigl‖|Dψ'(z_r)〉{\dot z}_r(z)-H|ψ'(z_r)〉\bigr‖².$$
If the system is a BEC, and the components $|ψ'(z_r)〉$ are coherent states with amplitudes $z_r$, this problem reduces to the Gross-Pitaevskii Equation.  Heller and others have investigated the problem for general systems with coherent state components, and found that the amplitudes obey something very like classical dynamics.  Therefore variational dynamics with sparse components is a waste of time: its results should be similar to those of the truncated Wigner method, and, if they differ, there won't be much reason to believe one rather than the other.

It is interesting to compare this to positive P simulations for the quartic oscillator.  These have predicted revivals, and have been done for large enough systems that the amplitudes of the coherent states must have been sparse.  I don't know whether both conditions were ever satisfied in the same simulation: it would be interesting to find out.

\beginsection{Modifications to Dirac-Frenkel}

The residual above can be defined using any norm one wishes.  The aim of computing quantum dynamics is usually to find some expectation values that could be compared to observed quantities, and one could consider using a norm that magnified the discrepencies in these observables, so that the solution with the least residual would correspond closely to the exact solution in the directions of interest.  On the other hand, we are computing quantum dynamics: these are unitary in the usual norm, so that a small error introduced at an early time is guaranteed to stay small when the state is propagated to later times.  No other norm guarantees this: they all introduce uncontrolled errors regarding how much early errors are magnified by later dynamics.

Equation~\ode\ will be integrated numerically.  For this to be done stably and efficiently, $\dot z$ must be a reasonably smooth function of $z$.  This is formally expressed in terms of the eigenvalues of the Jacobian $D\dot z$; different integration formulae place different bounds on the magnitudes and phases of these, but they all have limited regions of stability.  Moreover, as we will see below, a technically stable integration formula might still not produce a useful solution.

This complicates the solution of the problem in Formula~\residual.  It does not suffice to find a $\dot z$ with the smallest residual, for a given value of $z$.  In order for Equation~\ode\ to be numerically solvable, the properties of $\dot z$ under perturbation of $z$ are equally important.  Under the usual Hilbert space norm, Formula~\residual is a least squares problem; the pertubation properties of such problems are reasonably well understood.  We will examine ways to modify the problem, by introducing extra least-squares constraints, that might improve its perturbation properties.

\beginsection{Continuous phase space expansions}

The approximant studied in this paper has the form
$$|ψ(φ₁,α₁,⋯,φ_R,α_R)〉=∑_{r=1}^R e^{φ_r+α_ra†}|0〉
	=\pmatrix{|α₁〉&⋯&|α_R〉}\pmatrix{e^{φ₁+½|α₁|²}\cr\vdots\cr e^{φ_R+½|α_R|²}}
	=|A〉d,\tag\Ad$$
where $A=\pmatrix{α₁&⋯&α_R}\T $ is a vector of complex amplitudes, $|α_r〉$ is a coherent state, and $φ_r$ is a complex weight.  The form on the left is more complicated than the $|A〉d$ form, but it is an analytic function of $φ$ and $α$, which is convenient for derivatives and optimisation.  Wavelet theorists have studied the conditions on the set of amplitudes $A$ that allow any state to be expanded in this form, and the numerical stability of those expansions as the density of ampltiudes decreases towards the minimum for completeness.  This will be discussed later.

In the limit of large $R$, this sum approximates an integral over the amplitude $α$; any state can be approximated arbitrarily well by this form.  The continuous phase-space expansion is the topic of this and the next section.  Many problems in quantum optics have been solved by a family of Monte-Carlo methods, using similar expansions, where the amplitudes $α_r$ are propagated independently.  This gives some grounds to hope that the states of interesting many-particle systems can be expanded with a reasonable number of components $R$.

Expansion over coherent states is a linear operator from functions $f$ of a complex variable to quantum states $\G f$,
$$\G f=π^{-½}\int |α〉e^{-½|α|²}f(α*)\,d²α.$$
The notation $\int f(α)\,d²α$ means $\int_{-∞}^{∞}\int_{-∞}^{∞} f(x+iy)\,dx\,dy$.  The function $f$ is such that $e^{-½|α|²}f(α)$ is square-integrable.  Bargmann\cite{pam-14-187} showed that these functions form a Hilbert space under the inner product $〈g,f〉=\int e^{-|α|²}g*(α)f(α)\,d²α$; this paper refers to it as the space of phase space distributions, or just distributions.  When norms of functions are written below, the norm is the one derived from this inner product.

The state $|α〉$ is a coherent state with complex amplitude $α$.  These can be defined in a number of ways, perhaps most simply in second quantisation, as eigenstates of a boson annihilation operator $a$, with $a|α〉=α|α〉$.  The annihilation operator is not hermitian, so the eigenvalues $α$ are complex numbers.  In first quantisation, such an operator is the complex amplitude
$$a=(2\hbar)^{-½}(λx+ip/λ),$$
where $x$ and $p$ are the position and momentum operators for a particle with mass $m$.  The bosons in this case are vibrational quanta in a notional harmonic potential, centred at the origin, with frequency $ω=λ²/m$.  The coherent state $|α〉$ is a gaussian wave packet, with central position \dots and momentum \dots

Coherent states are not orthogonal, but their brackets follow the law
$$〈β|α〉=e^{-½|β|²-½|α|²+β*α}=e^{-½|β-α|²+i\Im α*β},\tag\csip$$
whence
$$\bigl|〈β|α〉\bigr|²=e^{-|β-α|²}.$$
When kets are considered as vectors, the angle between coherent states satisfies
$$\cos θ=‖|α〉‖‖|β〉‖\cos θ=|〈β|α〉|=e^{-½|β-α|²},$$
and comparing Taylor series gives
$$θ=|β-α|+O(|β-α|²).$$
This has a simple geometric analogy where phase space is mapped locally onto the surface of the unit sphere, the coherent states are vectors from the origin to the surface, and $|α-β|$ is a spherical angle.  The analogy fails when the angles between vectors are no longer small; for example, it is misleading in regard to the relation between $|α〉$ and $∂_α|α〉$.

The projector satisfies
$$\int |α〉〈α|\,d²α=π.$$
If the right hand side were 1, this would indicate that the coherent states were an orthonormal basis of ket space.  As it is, any state can be expanded as a combination of coherent states, in the form
$$|ψ〉=π^{-1}\int |α〉〈α|ψ〉\,d²α.\tag\csproj$$
However, this expansion is not unique: a coherent state $|β〉$ has the trivial expansion $|β〉=|α〉$, but the expansion given by Equations~\csproj\ and~\csip\ is a gaussian function of $α$.  A set of states like this is known as a frame, in distinction from a basis, whose elements are linearly independent, and whose expansions are unique.  Sets such as the coherent states are known as a Gabor frame.  Further properties of the coherent states will be stated as required.  Glauber's summary of them\cite{prx-131-2766} is paraphrased in every quantum optics textbook ever written.

Bargmann showed that the Hilbert space of wavefunctions is isomorphic to the space of phase space distributions, under the isometry $〈Γ⁺|$, where
$$〈 Γ⁺|ψ〉(α)=π^{-½}e^{½|α|²}〈 α|ψ〉.$$
Being a linear operator from quantum states to phase space functions, $〈Γ⁺|$ is written as a ket.  It is a pseudoinverse of $|Γ〉$, in the sense that $|Γ〉〈 Γ⁺|$ is the identity operator in state space.  The other composition, $〈 Γ⁺|Γ〉$, projects the functions of a complex variable onto the entire functions, orthogonally under the Bargmann inner product.  (Prove that, or find Bargmann's proof.)  It follows that $f=〈 Γ⁺|ψ〉$ is the function with least norm that satisfies $|Γ〉 f=|ψ〉$, with $‖f‖=\bigl‖ |ψ〉\bigr‖$.

If, following Bargmann, we identify state space with the entire functions, $|Γ〉$ is identified with the projector $〈 Γ⁺|Γ〉$.  Therefore, although $|Γ〉$ is singular, the singularity is easily resolved.  If we constrain the norm of $f$ to be not much greater than the norm of $|ψ〉$, which can be done simply enough by Tychonov regularisation, we ensure that $f$ is close to $〈 Γ⁺|ψ〉$.  And $〈 Γ⁺|$, an isometry, is as nice a function as could possibly be desired.  So it is curious that the discrete version of this problem has caused so much trouble when solved by least squares.

%% Consequences of Γ being a projection

The part of $|Γ〉$ that projects functions onto entire functions can be written as follows
$$π^{-½}〈β|Γ〉f=π^{-1}\int 〈β|α〉e^{-½|α|²}f(α*)\,d²α=π^{-1}\int 〈0|D(α-β)|0〉e^{-½|α|²}f(α*)\,d²α.$$
This appears to be a convolution of the function $e^{-½|α|²}f(α*)$ with the kernel $〈0|D(α)|0〉$.  There are many theorems to the effect that real convolution operators are badly conditioned.  It is interesting that these results do not apply to the complex convolution $|Γ〉$: as a projector, it is as nice an operator as it could possibly be.  I think the loophole is that of real functions can only oscillate at a limited rate, so that their Fourier transforms have finite norm.  Complex functions have space in the Argand plane to oscillate arbitrarily rapidy.  And the kernel $〈0|D(α)|0〉$ does: it is a wave packet with carrier frequency proportional to the imaginary part of $α$.


\beginsection{Frames}

Because we are after numerical methods, our goal is for expansions such as $\G f$ and $\A d$ to be numerically stable.  This means, for one thing, that a small change in $f$ should cause a small change in $\G f$.  The coherent states that compose $\G$ are overcomplete, so fixing the value of $\G f$ does not fix the value of $f$.  However, in order to compute with this expansion, we need to find a way of solving for $f$ such that a small change in $\G f$ causes a small change in the  $f$ that is the result of our method.

There is a precise way to state these conditions.  For the expansion $\G$ to be stable, its columns must form a frame\cite{1992-Daubechies-Ten}.  This means, in the continuous case, that there are constants $A$ and $B$ such that
$$ A‖|ψ〉‖²≤\int |〈γ|ψ〉|²\,dγ≤B‖|ψ〉‖²,$$
for all kets $|ψ〉$.  In the discrete case, the integral is replaced by a sum.  Because the operator $\G$ is linear, the ket $|ψ〉$ could be a difference $|φ₁〉-|φ₂〉$.  In this case, the right hand inequality guarantees that a small change in a ket will produce a small change in the components $〈γ|ψ〉$, which the left hand inequality says tht a small change in the components will produce a small change in the expanded ket.  

In the particular case of the coherent states, we have
$$ \int |〈γ|ψ〉|²\,dγ=〈ψ|\int |γ〉〈γ|\,d²γ\,|ψ〉=π,$$
so the frame condition is satisfied, with $A=B=π$.  A set of this kind, where the upper and lower frame bounds are the same, is known as a tight frame.  Mathematicians have known this for a long time, and refer to the coherent states, and other frames that result from translating and modulating functions that aren't necessarily Gaussian, as Gabor frames.

(Talk about dual frames, minimal norm expansions.)

In the numerical case, the number of frame kets is finite.  Unless $\A$ is singular, the frame bounds always exist, with $A$ and $B$ something like the largest and smallest singular values of $\A$.  So we need to play a bit loose with the mathematical definitions.  For numerical purposes, a bound of the order $10^{15}$ might as well be infinite, and one of the order $10^{-15}$ might as well be zero.  Also, a finite set of states can not  be a frame for the whole of Hilbert space: there will always be directions orthogonal to all the components, for which $∑_r|〈α_r|ψ〉|²=0$.  So we need to think carefully about what space $|ψ〉$ can lie in.  The obvious choice would be the range of $\A$, the span of the $|α_r〉$.

But this causes problems with the lower frame bound.  The amplitudes of the coherent states will be bounded: suppose $|α_r|≤a$.  A finite set of coherent states with different amplitudes are linearly independent, so the simultaneous equations $〈n|A〉d=0$, where n ranges from $0$ to $R-1$, have a solution.  This means that $\A d$ is a state of the form $c_R|R〉+c_{R+1}|R+1〉+⋯$.  Now, the component of $|n〉$ in $|a_r〉$ has modulus no greater than $a^n/\sqrt{n!}$, so $|c_n| ≤ a^n/\sqrt{n!} ‖c‖₁$.  This means that, in the limit of dense amplitudes, where $R$ increases while $a$ stays constant, we can find vectors $d$ such that $‖\A d‖$ tends to zero.  (Fix that to argue the right way round.)  There are states in the span of $\A$ with large particle number, that barely overlap any of the states $|α_r〉$, and have very small components along the frame directions.  So, from a numerical point of view, dense, bounded sets of coherent states fail to be frames.  If the $α_r$ extended over the whole of phase space, the set would include some larger amplitudes for which $〈α_r|ψ〉$ was not small.

The obvious solution is to not attempt to expand the entire span of $\A$, but first project onto some subspace for which the $|α_r〉$ are a frame with numerically reasonable bounds.  This is what regularisation methods do.  In particular, singular vectors of $\A$ with large singular values span a space for which the states are a frame.  In practice, these correspond to states with limited particle number.  So this might be why projecting onto smal l particle number makes simulations more stable: the ensemble of coherent states is a frame for that space!  It also suggest that, for accurate and stable expansions, we want coherent amplitudes large enough that the components don't have a great overlap with the state being expanded.

It is interesting to consider the squeezed states.  Mathematicians have thought about two types of frames: Gabor frames, generated by translation and momentum kicks, and wavelet frames, generated by translation and limited types of squeezing.  It is unclear whether anyone has studied if the squeezed states are a frame.  If they are not, it might explain why numerical expansions over squeezed states have proven to be so unstable.

Many of the above results for the operator $\G$ are true of expansions over frames in general.  We will treat the finite case, because every frame will eventually have to be represented on a computer.  So suppose the set of kets $|A〉=\pmatrix{|a₁〉&⋯&|a_R〉}$ is a frame for some subspace of Hilbert space.

The theory of frames is traditionally presented in terms of the frame operator,
$$S=∑_{r=1}^R |a_r〉〈a_r|={\rm Tr}(|A〉〈A|).$$
In the case that the columns of $|A〉$ are a frame, $S$ is invertible.  The set of vectors
$$|F〉=S^{-1}|A〉=\pmatrix{S^{-1}|a₁〉&⋯&S^{-1}|a_R〉}
	=\pmatrix{|φ₁〉&⋯&|φ_R〉}
$$
is also a frame, and is known as the dual frame of $|A〉$.  It has the property that $|A〉〈F|$ is the identity operator on the space.

For finite frames, expansions using dual kets have some special properties.  For a start, the dual expansion is the one where the vector of coefficients has the least norm.  In fact, if $|ψ〉=|A〉d$, and $c=〈F|ψ〉$, then $‖d‖²=‖c‖²+‖d-c‖²$\cite{1989-sir-31-626}.  This suggests that $d-c$ is forced to be orthogonal to $c$.

The methods will consider to approximate $|ψ〉=|A〉d$ set $d$ to a linear function of $|ψ〉$.  We will sometimes denote this mapping $d=〈A⁺|ψ〉$, because we intend that $|ψ〉≈|A〉d=|A〉〈A⁺|ψ〉$, whence $〈A⁺|$ is a pseudoinverse of the expansion operator $|A〉$.  We can then define an error operator $E=\bigl(1-|A〉〈A⁺|\bigr)$, so that $|ψ〉-|A〉d=E|ψ〉$.  Because $〈A⁺|$ is linear, it is equivalent to a column vector of bras $〈A⁺|=\pmatrix{〈φ₁|&⋯&〈φ_R|}\T ,$
such that $d_r=〈φ_r|ψ〉$.  We will call $|φ_r〉$ the dual ket of the component $|α_r〉$.  Note that the dual kets are not normalised.


\beginsection{Truncated phase space}

The operator $\G $ represents kets by phase space distributions, functions defined on the whole complex plane.  When the ket $|ψ〉$ has many particles, its smallest representation, the distribution $〈Γ⁺|ψ〉$, is concentrated far from the origin.  Such kets will occur when interacting particles are simulated, the interaction energy is proportional to the square of the number of particles, and the Hamiltonian operator amplifies many-particle components of the kets on which it acts.  In order to represent kets on a computer, the distribution functions must be sampled on a finite region of the complex plane.  If $〈Γ⁺|ψ〉$ extends beyond the sampled region, the ket can still be represented by another distribution, because $\G $ is not one-to-one.  However, this distribution will take larger values and vary more rapidly.  Therefore, if the simulation is to be numerically stable, it will have to be sampled carefully.

%% The SVE

This can be made quantitative.  Define a restricted operator
$$\Gx f=π^{-½}\int |α〉e^{-½|α|²}χ(α*)f(α*)\,d²α.$$
For example, $χ$ could be a characteristic function, with $χ(z)$ being 1 for $|z|≤b$ and 0 outside this disk.  For functions $f$ with support outside this disk, part of $f$ will be multiplied by zero, and the norm of $\Gx f$ will be reduced when compared to the norm of $f$.  This reduction would cause $\Gx$ to have some small eigenvalues, if it mapped functions onto functions or kets onto kets.  Because its domain and range are different spaces, it is necessary to use a generalisation, the singular values.

The idea of an eigenvalue expansion is to expand vectors over a carefully chosen basis, so that the effect of an operator is merely to rescale the components.  For the operator $\Gx$, there will need to be two bases, one in function space and one in ket space.  If it is required that the bases be orthonormal, and that $\Gx$ has the effect of rescaling the coefficients of vectors and replacing the function basis with the ket basis, the basis vectors are unique under similar conditions to those needed for uniqueness of eigenvalues.  The scaling factors are known as singular values; the singular value expansion of $\Gx $ has the form
$$\Gx f = ∑_{i=1}^∞ |u_i〉σ_i〈v_i,f〉.$$
The kets $|u_i〉$ are an orthonormal basis of state space, known as the left singular vectors of $\Gx$.  The $σ_i$ are positive real numbers, called singular values.  The functions $v_i$ are are orthonormal under the Bargmann inner product, and are called right singular vectors.

In the case where $χ(z)=1$, and $\Gx=\G$, all the singular values are 1, and the expansion is arbitrary.  The left singular kets $|u_i〉$ can be any orthonormal basis, and the right singular vectors are  $v_i=〈Γ⁺|u_i〉$.  This indicates that $\G$ is an incomplete projector.  The component of $|ψ〉$ in the null space of $\G$ is discarded, being orthogonal to every $v_i$, while the other component retains its expansion coefficients as the basis $v_i$ is replaced by $|u_i〉$.

When the operator is restricted, the SVE gives more information. If $\Gx$ were one to one, each $v_i$ would be the unique preimage of $|g_i〉$; since $\Gx$ is singular, $v_i$ is the preimage with least norm.  The singular values are the ratios of $‖\Gx v_i‖=‖σ_i|u_i〉‖$ to $‖v_i‖$.  If $|u_i〉$ has a low particle number, it can be expanded efficiently: $v_i$ will be close to $〈Γ⁺|u_i〉$, and $σ_i$ will be close to 1.  If $|u_i〉$ includes components with large particle number, its expansion will have a larger norm than itself.  In this case, $v_i$ will be a truncation of $〈Γ⁺|u_i〉$ to small amplitudes, multiplied by a large constant to maintain normalisation, and $σ_i$ will be small.

Before we study the singular values and vectors of $\Gx$, consider why small singular values are a problem.  The pseudoinverse of $\Gx$ with least norm is
$$〈Γ⁺_χ|=∑_{i=1}^∞ v_iσ_i^{-1}〈u_i|.$$  Suppose this is applied to a state $|ψ〉$ that has a low expected particle number, but small components with large particle number.  Such a component will overlap some $|u_i〉$ belonging to small $σ_i$, and $〈Γ⁺_χ|$ will multiply it by $σ_i^{-1}$, generating large components of those $v_i$ in the expansion function.  In the continuous expansion, this isn't a problem: the information is still there, and the state can be recovered by applying $\Gx$ to the expansion.  When the expansion is discretised, however, the small, low particle number component of $〈Γ⁺_χ|ψ〉$ that contains the important information about $|ψ〉$ is likely to be lost in the large component of $〈Γ⁺_χ|ψ〉$ that gives minor details of high particle number components of $|ψ〉$.

Let's compute some singular values and vectors.  Suppose that $χ$ is the characteristic function of a disk, with
$$χ(α*)=\cases{1 & $|α|≤b$ \cr 0 & otherwise.}$$
From symmetry, let's guess that the singular kets are the number states—we will check this numerically later.  It can be shown by a Lagrange multiplier argument that the right singular functions are the truncations to the disk of $f_n=〈Γ⁺|n〉$.  Then the singular values are
$$σ_n={‖\Gx f_n‖\over ‖f_n‖}.$$
We take
$$f_n=π^{-½}〈α|n〉e^{½|α|²}=π^{-½}∑_{n=0}^∞ {α^{n\ast}\over \sqrt{n!}}.$$

First evaluate $‖f_n‖$, using the Bargmann norm.  We have
$$\eqalign{ ‖f_n‖²&={1\over π}\int_{|α|≤b} e^{-|α|²}{|α|^{2n}\over n!}\,d²α \cr
	&={1\over π}\int_0^b e^{-r²}{r^{2n}\over n!}2πr\,dr \cr
	&={1\over n!}\int_0^{b²} u^ne^{-u}\,du=Γ(n+1,b²),}$$
where $Γ(n,x)$ is the incomplete gamma function, as defined in Matlab.  (Everyone defines this a different way.)  The other norm is given by
$$\eqalign{ \Gx f_n&=π^{-½}\int_{|α|≤b} |α〉e^{-½|α|²}f_n(α*)\,d²α \cr
	&={1\over π}∑_{m=0}^∞\int e^{-|α|²}{α^{m}\over \sqrt{m!}}|m〉{α^{n\ast}\over \sqrt{n!}}\,d²α \cr
	&={1\over π}\int_0^b dr re^{-r²}∑_{m=0}^∞{r^{m+n}\over \sqrt{m!n!}}|m〉\int_0^{2π} dφe^{i(m-n)φ} \cr
	&=\int_0^b dr 2re^{-r²}{r^{2n}\over n!}|n〉 \cr
	&=Γ(n+1,b²)|n〉.}$$

The singular value comes out to 
$$σ_n={‖ \Gx f_n‖\over ‖f_n‖}=\sqrt{Γ(n+1,b²)}.$$

Numerical experiments in the script {\tt csve.m} support this.  The operator $\G$ was discretised on a grid in phase space, with spacing $h=0.3$, bounded by the circle about the origin of radius $b$ for $b=3$, 4 and 5.  Ket space was discretised by taking inner products with the number states $|0〉$ through $|24〉$.  The graph below shows the singular values of the truncated operator $\Gx$.

%\topinsert \XeTeXpicfile svgrid.pdf width \hsize \endinsert

The contour plot shows how the analytic expression for $σ_n$ depends on $n$ and the truncation radius $b$.  The value has decayed to 0.7 when $n=b²$.  The rule of thumb is that we get one large singular value for every unit circle of phase space, with area $π$.

%\topinsert \XeTeXpicfile svcont.pdf width \hsize \endinsert

The following three figures show the right singular vectors, sampled versions of phase space functions.  The brightness of the plots indicates $|v_n(α)|$, the colors indicate phase.  These look very much like truncated number states.  As the grid expands, the pre-images of higher number states fit in it, and more singular values have converged.

%\topinsert \hbox{\XeTeXpicfile rsv3.pdf width 0.5\hsize
%\XeTeXpicfile rsv4.pdf width 0.5\hsize}
%\XeTeXpicfile rsv5.pdf width 0.5\hsize\hfil\break
%The right singular functions $v_i$ for the truncated expansion operator $\Gx$, where phase space is truncated at $|α|=3$, 4 and 5.  These are functions of a complex variable: the brightness of the plot indicates modulus, and the color indicates phase.\endinsert

Reducing the dimension of Fock space so that the problem is underdetermined means that, when the problem is solved by SVD, we only take components along $N+1$ directions in Fock space, and map them to $N+1$ directions in parameter space.  The remaining $2R-N-1$ components of the solution are zero.  As we add extra dimensions to Fock space, we are taking components in more Fock space directions, which include more oscillating components of the wavefunction.  This is like sampling a function on a finer grid.  With these directions, we can form some of the smaller singular vectors of the full problem.

\beginsection{Discrete phase space expansions}

As mentioned above, the computer manipulates a discrete expansion with the form of Equation~\Ad,
$$|ψ〉=d₁|α₁〉+⋯+d_R|α_R〉=|A〉d.$$
This kind of expansion has been studied for a long time.  An early question was how dense the amplitudes $α_r$ must be in order for the set to span all kets, and $|A〉$ to be a frame.  In the case that the amplidues are a rectangular grid, with $α_r=ma+inb$ Von~Neumann conjectured in 193? that the condition is one amplitude per area $π$ of phase space, which corresponds to an area $\hbar$ of the $x-p$ plane.  This was proved in 19?.  The conditioning of expansions as the density of sampled amplitudes approaches this limit appears to be an open question.  Another open question is how much this limit can be relaxed if squeezed states are included in the expansion.

In recent decades, these expansions have been studied intensely by engineers involved in signal processing, who call the set of coherent states a Gabor~frame.  These researchers regard gaussian wavepackets as a kind of time-limited Fourier transform, with a gaussian window function, following on from the work of Hamming and others at Bell Labs.  Gabor frames were investigated by the same community that developed wavelets; the latter became of very wide interest, and obscured the earlier work.

If the system has only a few modes, so that $α$ is a low-dimensional vector, the set of sampled amplitudes $A=\{α₁, …,α_R\}$ could be a regular grid, with $α_r=(m+in)h$, $m$ and $n$ being integers.  This makes many things easier.  In particular, it is straightforward to relate the weights $d$ to the values of a continuous distribution $f$ at the points $α_r$.  It is convenient to normalise so that $‖d‖=‖f‖$.  If the $α_r$ form a grid with spacing $h$, as specified above, and we set $d_r=cf(α_r)$, $c$ being constant, this normalisation condition is satisfied when
$$∑_r|d_r|²=c²∑_r|f(α_r)|²=\int |f|²\,d²α≈h²∑_r|f(α_r)|².$$
So the proportionality constant is $c=h$, and the discretisation is
$$d_r=hf(α_r).$$

Relating $d$ to a continuous function is harder when the amplitudes $A$ are irregular.  The amplitudes will not generally be drawn from a uniform distribution—if they were, we would no doubt be better off using a sparse grid—and if amplitudes distributed with probability $P(α)$ are assigned weights proportional to $d_r=f(α_r)$, then, in the limit of many samples, the state $|ψ〉$ will converge to a multiple of $|Γ〉Pf$, with the continuous distribution $f$ scaled by the distribution $P$ of the sample amplitudes.   This convergence will be slow, presumably with the square root of $R$, and we might hope to speed it up by compensating for the random spacing of the sample amplitudes.  If many coherent states overlap $|α_r〉$, then values close to $f(α_r)$ will be counted many times in the sum of Equation~\Ad.  In order for the sampling to remain even, the ratio of $d_r$ to $f(α_r)$ ought to depend on the density of $A$ around $α_r$, with smaller $d_r$ where the amplitude samples turn out to be dense.

The normal way to discretise a linear operator such as $|Γ〉$ is to select a set of functions $u_r$ over which to expand $f$, and a set of kets $|v_r〉$ over which to expand $|ψ〉$.  Given a set of sample amplitudes $A$, the only obvious choice for $|v_r〉$ is the set $|A〉$.  There are two apparent choices for $u_r$: delta functions $δ(α-α_r)$, or the gaussian functions $〈Γ⁺|α_r〉$.  Operators are usually discretised by the Galerkin condition, where $f$ is approximated by $Ub$, the orthogonal projection of $f$ onto the span of the $u_r$, the discrete approximation to $|ψ〉$ is $|V〉d=|V〉Tb$, for some matrix $T$, and it is required that $|V〉d$ is the orthogonal projection of $|Γ〉Ub$ onto the span of the kets $|v_r〉$.  If we choose delta functions, the projections are trivial: we're sampling $f$ at the points $α_r$, and adding up coherent states with the resultant weights.  The expansion of $f$ over Gaussians is more complicated: we are effectively setting $|ψ〉$ to the projection of $|Γ〉f$ onto the span of the $|α_r〉$.  So it seems that orthogonal projections onto sets of Gaussian functions are an essential part of this discretisation.

There are two usual ways to relate a continuous function $f$ to a discrete set of samples $d$.  This is usually done in one of two ways: by a vector
$$\vec g=\pmatrix{g(z₁) & ⋯ & g(z_R)},$$
where the points $z_i$ form a regular grid, or by
$$\vec g=\pmatrix{〈 u₁,g〉 & ⋯ & 〈 u_R,g〉},$$
where the distributions $u_i$ are a fixed basis.  The huge dimension of state space rules out any regular approach: we must sample $g$ at a set of irregularly spaced amplitudes or on an irregular basis, adapted to the support of $g$.

\hrule

This expansion is formally equivalent to the continuous one, in that
$$∑_r d_r|α_r〉=|Γ〉f_A,$$
where
$$f_A(β)=π^½∑_rd_re^{½|α_r|²}δ(β-α_r).$$
I don't know enough about generalised functions to reliably manipulate these forms; in particular, the norm of $f_A$ is a bit mysterious.  If this could be clarified, it might provide some useful insights, for example how the norm of $f_A$ compares to that of $〈Γ⁺|ψ〉$.

To get a sense of what residuals to expect from optimised expansions, the expansion of a state as an entire function, $f=〈Γ⁺|ψ〉$, can be sampled at discrete amplitudes, and the integral $|Γ〉f$ evaluated as a sum.  In the limit of many amplitudes, this should converge to $|Ψ〉$.  I chose the vacuum state as $|Ψ〉$.  I evaluated the integral in two ways, supposed to be extreme cases.

The first way used a grid, restricted to the disk $|α|<2.5$.  The pitch of the grid, $h$, was varied, and $f(α_r)=π^{-½}e^{½|α_r|²}〈α_r|ψ〉$ evaluated at each point.  The integral was discretised in the obvious way, to give approximations
$$|ψ_R〉={1\over πh²} ∑_{r=1}^R |α_r〉〈α_r|ψ〉.$$
A residual $‖|ψ〉-|ψ_R〉‖$ was found for each $R$; these are plotted as the solid line in the figure.  As the components increase, the residual asymptotes to the phase space truncation error $‖|Γ_χ〉〈Γ⁺|ψ〉-|ψ〉‖$.  (Calculate that precisely.)  The gaussian function $f$ is very smooth, so this converges rapidly as $R$ increases.

The second way was a type of stochastic integration.  The samples $α_r$ were drawn randomly, from the normal distribution $|〈α|ψ〉|$.  The integral was approximated by 
$$|ψ'_R〉=∑_{r=1}^R |α_r〉{〈α_r|ψ〉\over |〈α_r|ψ〉|},\qquad |ψ_R〉={|ψ'_R〉\over||ψ'_R〉|}.$$
The phase of $f$ is integrated by sampling, but the modulus is integrated stochastically.  The residuals are plotted as the broken line.  In principle, this will converge exactly as $R$ increases.  However, the convergence is $R^{-½}$, as seen on the graph, and expected from the laws of large numbers.

These two lines set bounds on how well we can reasonably expect to solve the expansion problem on an irregular grid.  A least squares method that does no better than sampling the argument of $f(α)$ at random $α$ is worthless: we need to do better than the dashed line.  On the other hand, it seems unreasonable to expect samples on an irregular grid to fit $|Ψ〉$ better than samples on a square grid.  If we try to match the solid line, we should expect the expansion coefficients to be very sensitive to the grid, and not to resemble $f(α_r)$.  If we try to do better than the truncation error, we will have to represent large number components, and the expansion coefficients will become very large.

%\topinsert\XeTeXpicfile converge.pdf width \hsize
Convergence of a sampled vacuum state, as a function of the number of sample amplitudes.  The solid line shows sampling on a regular grid, the dashed one at random amplitudes.  Details in text.  The dotted line shows the difference between the expanded state and a vacuum state truncated to the extent of the grid; the norm of the truncation error is around $10^{-3}$.  The tension is the Frobenius norm of the matrix $T$ defined in the text.
%\endinsert

The problem of expanding a state over coherent states is similar to expanding a vector over an overcomplete basis in the plane.  The main difference is that large sets of vectors in the plane are linearly dependent, while sets of coherent states are not.  The two approaches are illustrated below, with an expansion of the vector marked $+$ along the indicated directions.  The black lines show the expansion
$$ \vec x = π^{-1}∑_{r=1}^R (\vec u_r\cdot\vec x)\vec u.$$
As can be seen, this is a very good approximation even with five vectors; in fact, it is hard to visually distinguish the approximate point from $\vec x$ even with 3 directions.  The red lines show $\vec x$ expanded over the two directions closest to it.  The interesting thing is the size of the change of the expansion coefficients when $\vec x$ moves perpendicular to $\vec x$.

%\topinsert\XeTeXpicfile circle-1.pdf \endinsert

\hrule

The obvious way to attempt this is to factor the coefficients as $|ψ〉=|A〉 Wc'$, where $c'_i$ are samples that approximate $g(α_i)$, and $W$ is a matrix of weights that adapts the samples to the irregular grid.  In fact, we can do this exactly.  If we set $W=(〈 A|A〉)^{-1}$, then $c'=〈 A|ψ〉$.  We could try seeking coefficients $c$ such that the norm of $〈 A|A〉 c$ is minimal.

To avoid the complexities of Wirtinger calculus, the problem of relating the weights of gaussians to the values of a sampled function can be set up for a function of a single real variable.  It would be appealing for the weight matrix $W$ to be diagonal, so that the weight of the gaussian centred at each point was determined by the value of $f$ at that point.  This can be set up either as a Galerkin condition, or as a collocation condition: they turn out to be equivalent.  Unfortunately, the weights end up oscillating wildly.  This approach doesn't seem useful, but is presented for completeness, and because the phenomenon of adjacent weights cancelling keeps occuring, this being the simplest problem in which I've seen it.

As a collocation problem, consider the case where $f(x)=1$, and we wish to approximate $f$ by a function 
$$f'(x)=∑_i e^{-(x-x_i)²}w_ic_i$$
that satisfies the collocation condition $f'(x_i)=f(x_i)$.  The goal is for $c_i=1$, so we need ...

$$\pmatrix{1 & e^{-|α₁-α₂|²} &  & e^{-|α₁-α_R|²} \cr
	e^{-|α₂-α₁|²} & 1 &  & e^{-|α₂-α_R|²} \cr
	& & \ddots & \cr
	e^{-|α_R-α₁|²} & e^{-|α_R-α₂|²} & & 1}
\pmatrix{w₁²\cr w₂²\cr\vdots\cr w_R²}
= \pmatrix{π\cr π\cr\vdots\cr π}.$$

The Galerkin condition seeks a set of weights $w$ such that, for the normalised functions $u_i=π^{-½}e^{-(x-x_i)²}$, we have
$$1=\int f\cdot u_i=∑_j w_j u_i(x_j)=∑_j w_je^{-(x_j-x_i)²}.$$
This is the same condition as the collocation condition.  The results are shown below.

Unfortunately, for random samples $A$, there tends not to be a solution where the weight matrix is diagonal: some of the $w_r²$ would have to be negative.  It looks like the amplitudes on the edges of the grid need large weights, because half of their corresponding gaussian lies outside the grid, and isn't sampled.  The gaussians centred inside the grid are sampled at all these points, and need negative weights at the inside points to compensate.  This sounds like the problem that Chebyshev points solve, where the function is sampled more closely at the edges of the domain.

There is another way to derive coefficients of the constant function, which might be more stable.  Suppose that $D$ is a subset of the real line, large enough that the tails of gaussians are neglibible outside it.  Let 
$$U=π^{-½}\pmatrix{e^{-(x-x₁)²}&⋯&e^{-(x-x_R)²}},$$
and $w$ be a column vector of $R$ real weights.  Then define a residual
$$r(w)=\int_D(Uw-1)².$$
The derivative of this with respect to the weights is
$$Dr(w)=\int_D 2(Uw-1)U=2\int_D U\T Uw-U.$$
(Note that, for row vectors $a$ and $c$, and $b$ a column vector, $abc=c\T ab$.)
Unfortunately,
$$\int (U\T U)_{ij}=π^{-1}\int e^{-(x-x_i)²-(x-x_j)²}\,dx
	={1\over\sqrt{2π}}e^{(x_i-x_j)²/2},$$
and $\int u_i=1$, so this is the identical problem again.

An obvious way would be to assign amplitudes to the nearest sample.  That makes the volume for each sample a quite irregular prism; maybe there is a simple way to calculate the volumes of those, but I don't know it.  What if we assign a value to each point as an average over the sampled values, weighted, say, in proportion to the squared distances?  There would need to be some extra condition to ensure the function tends to zero at points remote from the samples.  Something like $x^{-2}e^{-|x|²}$ sounds plausible: it's quadratic at sort range, where the gaussian is flat, but outside the support of the samples, it decays in the right way for a phase space expansion.  I think the short range part needs to be a power law, so that it doesn't have a length scale built in when the distances between samples are random.  What's the best exponent?  I expect it depends on the dimension of the phase space, which would determine how many samples we expect to find close to a given one.

To prevent the weights from oscillating in order to fit the edges of the samples, it is tempting to do a weighted least squares fit, which emphasises fitting the centre of the cluster over fitting the edges.  There are two ways to do this.  Firstly, instead of expanding the constant function $f(x)=1$, we can expand a peaked function such as $f(x)=e^{-x²}$, then divide the coefficient of $e^{-(x-x_r)²}$ by $e^{-(x_r)²}$ to get a weight.  This mimics the expansions we need to do in practice, where the sample points cover the region where the expanded function has large values.  The residual in this case is
$${\cal E} = \int\left(e^{-x²}-∑_rd_re^{(x-x_r)²}\right)²;$$
the only change to the normal equations is that the right hand side $\pmatrix{1&⋯&1}\T $ is replaced by a constant multiple of 
$$b=\pmatrix{e^{-x₁²/2}&⋯&e^{-x_R²/2}}.\tag\rhs$$
For now, I'm ignoring factors of 2, $\sqrt π $ and so on: how the weights oscillate is more interesting than their size.  This seems to work: with random sample points, the weights still oscillate, but remain within an order of magnitude or two, where the weights from expanding the constant function vary by a factor of a million.

The other approach is a weighted least squares fit of the constant function, where the residual is 
$${\cal E} = \int e^{-x²}\left(1-∑_rd_re^{(x-x_r)²}\right)².$$
The right hand side of the normal equations remains that of Equation~\rhs, and the matrix becomes
$$a_{ij}=e^{-(x_i²+x_j²-6x_ix_j)/4}.$$
(I'm still ignoring constants of order 1.)  Surprisingly, this doesn't seem to work any better than the unweighted problem.


%% Sampling on random grids

%\topinsert \XeTeXpicfile gweight.pdf width \hsize 
The weights $w_i$ required in order that the integrals of the functions $u_i(x)=e^{-(x-x_i)²}$ are given by the sums $∑_j w_j u_i(x_j)$.  The points $x_i$ are normally distributed (top), or evenly spaced (bottom).  Note how nearby points tend to cancel each others' samples.  I expect experts in Chebyshev sampling would not be surprised by this, but I am not yet such an expert.

The problem with the evenly spaced points is the gaussians at the edge of the grid: only half their support is covered by the grid, so that half needs to be doubly weighted.  This would overweight the gaussians just inside the grid, so the only solution is an oscillating one.  Similar things happen in the random ensemble, at the edges of a subgrid of closely spaced points.  Nearly the same thing happens when sampling with coherent states: the extreme amplitudes need extra weight, which must be cancelled by the interior ones.  When states are sampled on regular grids of coherent states, the outside amplitudes have very little weight in any case, and don't affect the solution much.  This only happens when the cloud of sampling amplitudes extends beyond the support of the distribution, and the outer amplitudes have very small weights.  So one solution  would be to find the best expansion of $e^{-x²}$, instead of 1, then adjust the weight for $x_i$ by either $e^{-x_i²}$ or $〈e^{-(x-x_i)²},e^{-x²}〉$.
%\endinsert

The goal of fiding samples of the phase space distribution $f(α_r)$ is to regularise the expansion and inverse problems.  The expansion function with the least norm is guaranteed to be a nice function: we don't know much about the finite superposition whose weights have the least norm.

To the the norm of the function $f$, sampled on an irregular grid $α_r$, the obvious method is to interpolate.  Doing this might help with irregular weights due to irregular sample points, because a least-squares interpolant will average the weights at several adjacent points.  However, there is probably a better way to adjust the weights for irregular sampling.

We want to interpolate a function of a complex variable, that should be nearly analytic.  To me, the obvious way to do it is a least squares fit to the monomials
$$U=e^{-½|z|²}\pmatrix{1 & z & z* & z² & zz* & z^{2\ast} & z³ & z²z* & ⋯}.$$
It helps to think of $z^mz^{n\ast}=r^{m+n}e^{i(m-n)φ}$.  Note that if these monomials are orthogonalised by Gramm-Schmidt, in the order they are written, the functions $e^{-½|z|²}\pmatrix{1 & z & z² & z³ &  ⋯}$ are retained as part of the orthonormal set: $z^n$ is the first monomial proportional to $e^{inφ}$, so it is orthogonal to all those before it.  The same is true for $z^{n\ast}$, the first monomial proportional to $e^{-inφ}$.  So this basis is an obvious one for projecting onto analytic functions in $z$ or $z*$.  It will have the form
$$e^{-½|z|²}\pmatrix{1 & re^{iφ} & re^{-iφ} & r²e^{2iφ} & p_{11}(r) & r²e^{-2iφ} & r³e^{3iφ} & p_{12}(r)e^{iφ} & p_{21}(r)e^{-iφ}& ⋯},$$
where $p_{ij}$ is a polynomial of degree $i+j$.  Someone must have investigated these orthogonal polynomials before, but they don't appear in Courant \& Hilbert or in Abramowitz \& Stegun.

For least squares fitting, orthogonal polynomials are not necessary: the QR factorisation does that work for us.  To find norms, we do need to know the grammian matrix, whose elements are the Bargmann inner products of the monomials:
$$\eqalign{a_{mnpq}&=\int (z^{m\ast}zⁿ)*(z^{p\ast}z^q)e^{-|z|²}\,d²z \cr
	&= 2πδ_{m+n,p+q}\int_0^∞r^{m+n+p+q+1}\,dr \cr
	&= πδ_{m+n,p+q}\int_0^∞r^{m+n+p+q}\,2r\,dr \cr
	&= πδ_{m+n,p+q}Γ({m+n+p+q\over 2}+1)=π(n+p)!δ_{m+n,p+q}.}$$
Some fits of these to expansions of cat and coherent states over random amplitudes are shown below.  They suffer the usual problem that polynomial fits oscillate more wildly as the degree of the polynomial increases; there are meant to be ways to avoid that, but I'm not an expert in them.

\beginsection{Regularisation of the discrete expansion problem}
%% Picard plots

The most basic issue is the Picard condition.  This restates a simple observation: for the problem $\G \dot f=H|ψ〉$ to have a solution $\dot f$, the expansion of $\dot f$ over an orthonormal basis must have a finite norm.  If the basis is chosen as the right singular vectors $v_i$, this means that the sum of the squared moduli of their coefficients must be finite.  The component of $v_i$ is $〈u_i|H|ψ〉/σ_i$.  So the Picard condition is that, for the equation to have a solution, the series
$$∑_{i=1}^∞{\bigl|〈g_i|H|ψ〉\bigr|²\over σ_i²}$$
must converge.  The singular values reduce with $j$, by definition; this requires that the components of the left singular vectors reduce somewhat faster.

The variational problem seems to be pretty well-posed: if the initial superposition is the phase space function sampled on a grid, then the state that comes from a variational change in the amplitudes matches $H|ψ〉$ very closely.

%% Expansion tension

To capture the concept of a smooth expansion over an irregular set of nonorthogonal vectors, we'll define a quantity called the sampling tension.  Suppose that $|u〉$ and $|v〉$ are nearly parallel vectors, and the state $|ψ〉$ is expanded over a set that includes them, as $|ψ〉≈c_u|u〉+c_v|v〉+⋯$.  A relaxed expansion is one where $c_u|u〉$ and $c_v|v〉$ are as similar as possible; this can be ensured, regardless whether the $c$ are complex numbers of whether $|u〉$ and $|v〉$ are parallel or antiparallel, by requiring that the component of $c_u|u〉$ along $|v〉$ is equal to the component of $c_v|v〉$ along $|u〉$.  The difference between these components is
$$T_{uv}=\left| {c_v〈u|v〉\over ‖|u〉‖} - {c_u〈v|u〉\over ‖|v〉‖}\right|.$$
These differences form a tension matrix.  In the case that $|u〉$ and $|v〉$ are nearly orthogonal, their components should be independent.  In this case, the $T_{uv}$ will be small, because of the proportionality to $〈u|v〉$.  The tension matrix is a reasonable measure of the smoothness of an expansion: its Frobenius norm, the sum in quadrature of the components, would be a good summary.

The idea is that a relaxed expansion has all the coefficients pulling in the same direction, but an expansion with large tension is a tug-of-war, with adjacent coefficients pulling the expanded state in opposite ways.

The tensions were calculated for the grid and irregular expansions above, and plotted in the second graph.  Despite the much worse convergence of random sampling, the weights have very nearly the same tension as the regular grid.  

\beginsection{Tychonov conditioning}

The simplest approximation method is Tychonoff regularised least squares, with different values of the regularisation parameter $λ$.  This assigns expansion coefficients $d$ in order that
$$\pmatrix{|A〉\cr λI}d≈\pmatrix{|ψ〉\cr 0},$$
this equation being satisfied in a least-squares sense.  The solution minimises the residual $‖|ψ〉-|A〉d‖²+λ²‖d‖²$.  It is given by
$$d=〈A⁺|ψ〉=\pmatrix{|A〉\cr λI}⁺\pmatrix{1\cr 0_R}|ψ〉,$$
where $()⁺$ denotes the Moore-Penrose pseudoinverse of a matrix, $I$ is the $R×R$ unit matrix, and $0_R$ is a $R×1$ zero vector.  I.e., the vector $〈A⁺|$ is the leftmost column of the pseudoinverse matrix.

Suppose $|A〉$ has the singular-value decomposition $|A〉=|U〉SV†$, so that
$$\pmatrix{|A〉\cr λI}v_r=\pmatrix{σ_r|u_r〉\cr λv_r}=\sqrt{σ_r²+λ²}w_r.$$
The orthonormality of the $w_r$ follows from that of the $|u_r〉$ and $v_r$, so we have a singular-value decomposition
$$\pmatrix{|A〉\cr λI}=WTV†,$$
where $τ_r=\sqrt{σ_r²+λ²}$.  Then 
$$〈A⁺|=\pmatrix{|A〉\cr λI}⁺\pmatrix{1\cr 0_R}=VT^{-1}W†\pmatrix{1\cr 0_R},$$
where the last two factors are the first column of $W†$,
$$W†\pmatrix{1\cr 0_R}=\pmatrix{{σ₁\over\sqrt{σ₁²+λ²}}〈u₁|\cr\vdots\cr 
	{σ_R\over\sqrt{σ_R²+λ²}}〈u_R|}.$$
It follows that the right singular vectors of $〈A⁺|$ are the bras $〈u_r|$, and the singular values are 
$$σ⁺_r={σ_r\over τ_r\sqrt{σ_r²+λ²}}={σ_r\over σ_r²+λ²}.$$
Tychonov regularisation divides the singular values of $|A〉$ into large ones and small ones, as compared to $λ$.  For large $σ_r$, the inverse is very close to the unregularised problem, with $σ⁺_r≈1/σ_r$.  For small $σ_r$, the component in that singular direction is nearly discarded by $|Α⁺〉$, with $σ⁺_r≈σ_r/λ²$.  Note that, in this limit, $|Α⁺〉=λ^{-2}|A〉$!  In between, there is a maximum at $σ_r=λ$, with $σ_r=1/2λ$, half the unregularised value.

This has dramatic consequences for the conditioning of the regularised problem, and how that varies with $λ$.  When $λ<σ_R$, regularisation has very little effect: the condition of $〈Α⁺|$ is the same as that of $|A〉$, which, for dense sets of coherent states, is usually dreadful.  As $λ$ increases, so that $σ₁<λ<σ_R$, the singular values lie on either side of the $σ_r=λ$ peak; the condition of $〈Α⁺|$ is determined by how far below this peak $σ₁$ and $σ_R$ fall, and the condition number has a sharp minimum when $σ⁺₁=σ⁺_R$.  When $λ$ dominates the singular values, and $|Α⁺〉=λ^{-2}|A〉$, it is again the case that their condition is the same.

\midinsert %\XeTeXpicfile tysv.pdf width \hsize
Figure 3 \endinsert

This is illustrated in Figure~3, which plots the logarithms of the singular values of $|A〉$ against those of $|A⁺〉$, both scaled to the regularisation parameter $λ$.  In the right hand plot, $λ$ is small compared the the singular values, so these lie very close to the line $σ⁺=1/σ$.  In the left hand plot, $λ$ is large, $|A⁺〉$ is nearly $λ^{-2}|A〉$, and the singular values lie on $λσ⁺=σ/λ$.  The singular values have been inverted, but this does not change the ratio of the smallest to the largest, so in both cases the condition numbers of $|A〉$ and $|A⁺〉$ are the same.

In the middle plot, $λ$ is of the same order as the singular values, which is the usual case in well-regularised problems.  In the illustrated case, where the singular values span several orders of magnitude, the condition number of $|A⁺〉$ is about the square root of that of $|A〉$.  If the singular values were more even to begin with, and all fit in the curve at the top of the peak, the improvement would be greater.  However, in this case it is unlikely that the problem would need to be regularised, so the square root improvement is likely the best that can be expected.

The leftmost column of the figure illustrates a large regularisation parameter.  The norm of $d$ is constrained to be small, so the approximation is very bad.  However, the approximation coefficients are very well behaved: they are brackets of $|ψ〉$ with coherent states.  The rightmost column illustrates a small regularisation parameter, where $d$ is essentially the least squares solution to $|A〉d=|ψ〉$.  This approximation is good, and the plot of the error operator is mostly black.  In fact, it is surprisingly good: the largest component amplitude is around 2, but the approximation is nearly exact up to $n=11$.  There is a price to pay for this.  The dual kets have large norms, and they are close to the number state $|n=14〉$.  This means that approximations of states with small amplitudes not be stable: perturbing such a state by adding a small component of $|n=14〉$ will wreak havoc with its expansion coefficients.  Using the approximation method of the leftmost column, this component would have very little effect on the coefficients.  This type of instability is very common in least squares problems, and the purpose of regularisation methods such as Tychonov's is to avoid it by trading off the ability to represent large number states.  The middle columns show some intermediate values of $λ$.  In the third column, the dual kets are close to 1, and number states up to $n=4$ or 5 are approximated well, as would be expected for an expansion over amplitude 2 coherent states.  This seems to achieve the right tradeoff.

There is a family of methods for inverting operators with small singular values, known as regularisation methods.  A very useful idea is due to Tychonov.  The problem
$$\Gx f=|ψ〉$$
can be regarded as a least squares problem, where $|ψ〉$ is chosen to minimise the residual $‖\Gx f-|ψ〉‖²$.  The exact solution has residual zero, of course.  The problem is that it is dominated by components belonging to singular values of $\Gx$.  Suppose we modify this problem, and seek $|ψ〉$ to minimise $‖\Gx f-|ψ〉‖²+λ‖f‖²$, where $λ$ is a positive number, known as a Lagrange parameter for reasons that will become clear shortly.  There are two things to note here.  Firstly, this is still a least squares problem: it is equivalent to minimising 
$$\left‖ \pmatrix{\Gx \cr λ} f-\pmatrix{|ψ〉\cr 0}\right‖².$$
Therefore, it can be solved stably by the usual least squares methods.  Secondly, suppose $f_λ$ is the solution for a given value of $λ$.  If another function, $g$, had the same norm but a smaller residual, then $f$ would not minimise the Tychonov expression.  The Tychonov method is equivalent to constraining the norm of $f$, and seeking the solution with smallest residual given the constraint.  Or, equivalently, constraining the residual and seeking the solution with smallest norm.  This equivalence to a constrained optimisation problem is why $λ$ is a Lagrange parameter.

Solving the Tychonov problem for various values of $λ$ generates a family of solutions.  These can be plotted on a graph of residual against solution norm, where they mark the edge of the space of possible solutions.  Above and to the right of the curve, a solution can be found with the given norm and residual: below and to the left, it can't.

%\topinsert\XeTeXpicfile lcurve-1.pdf\vskip 5mm\noindent\it L curve for a typical application of Tychonov conditioning\endinsert

An example for a typical application is shown above.  The context in which Tychonov regularisation is usually applied is that some observations $b$ are predicted by a physical state $x$, according to the linear law $Ax=b$.  The real state is $x₀$, and the observed data are $b=Ax₀+e$, where $e$ consists of measurement error.  The problem $Ax=b$ is solved for $x$, and regularised with various Lagrange parameters.  At the top left, the norm of $x$ is constrained to be zero, so the solution is $x=0$, and the residual is the norm of $b$.  As the constraint on the norm is relaxed, the solution becomes closer to $x₀$, and the residual decreases.  At the elbow of the curve, the norm of the solution is constrained to be close to the norm of the real state $x₀$, the solution is close to $x₀$, and the residual is the norm of the measurement error.  Relaxing the constraint further allows spurious solutions $x$ that fit the measurements better than the real state $x₀$ does.  Typically, they won't fit much better, so the residual doesn't decrease much below $‖e‖$.  However, the solution now includes components belonging to small singular values, and the norm of the solution will increase as far as it is allowed to.

%% ill-posedness

As shown above, the restricted expansion operator $\Gx $ has singular values that decay to very small values, whose singular vectors are states with particle numbers corresponding to coherent amplitudes outside the support of $χ$.  

There are two types of ill-posed problems, rank-deficient and discrete.  They are distingished by how the singular values decay to zero.  In a rank-deficient problem, there is a sharp cut between large and small singular values.  The large ones are large enough that components in their singular directions do not contribute excessively to the solution, the small ones would contribute huge components, but they are small enough to round to zero, and ignore the component along that singular direction.  In discrete problems, the singular values decay continuously.  The size of the components of the solution varies continuously: there is no point where they become huge, but a continuous minimum and a gradual rise as the singular values get smaller.

I've been assuming that the variational problem is discrete, because it comes from a discretisation of an integral equation.  However, the integral equation is a complex convolution, and for some values of the complex amplitude, the kernel oscillates arbitrarily sharply.  This is the same sort of integral equation as the Fourier transform, where we know $f(t)$, and want to find $F(ω)$ such that
$$f(t)=\int F(ω)e^{iωt}\,dω.$$
This also has a kernel that oscillates at a rate proportional to $ω$, and it also has very stable solutions.

A numerical example is shown below.  The state $|ψ〉$ is the Schrödinger cat that a coherent state becomes after half a cycle in a quartic oscillator; the amplitude of the initial coherent state was $2$.  A set of 40 sample amplitudes $α_r$ was drawn from the probability distribution $P(α)=|e^{-½|α|²}〈G⁺|ψ〉(α*)|²$.  These are plotted below, as circles in phase space.  For ensembles with fewer components, the results were very inconsistent between draws.

The state $|ψ〉$ was expanded in the form $|ψ〉=|A〉\vec f$, where $|A〉=\pmatrix{|α₁〉&⋯&|α_R〉}$, and $\vec f$ is a column vector of $R$ complex weights.  This was done with Tychonov regularisation, for various Lagrange parameters: these and the residuals are shown with the graphs.  The complex weights $c$ are indicated by brightness and color as above.  Note that least norm regularisation has the desirable property that nearby $α_r$ have similar $c_r$: the least norm solution to $a+b+c=X$ is $a=b=c=X/3$.

%\topinsert\XeTeXpicfile cats.pdf width \hsize
%\vskip 5mm\it A cat state expanded over a random set of 40 amplitudes, as described in the text.\endinsert

A basic question is why regularisation helps.  Usually, it's applied to the inversion of measured data, where the right hand side of the least squares problem includes noise.  The aim is to prevent amplified noise from dominating the solution.  In our problem, the right hand side is calculated instead of measured, and the only noise is rounding errors.  Those are 16 orders of magnitude smaller than the exact right hand side, so rounding the small singular values up to $10^{-10}$ should be more than adequate.  We've needed much stronger regularisation than that: therefore, rounding errors aren't the problem.

The only other error I can think of is nonlinearity.  Tychonov parameters of order $10^{-10}$ do work for the linear expansion problem.  Perhaps we're not doing regularisation, we're actually establishing a trust region, where the linear least squares problem is a good approximation to the actual change in $|ψ(z)〉$.

If that's the case, it's tempting to make the weights linear instead of logarithmic, so that that part of the problem is exactly linear.  However, the function $e^φ$ is pretty nearly linear in the range we're shifting $φ$, so that can't help much.  In fact, the whole problem is quite linear.

To avoid the amplitudes clustering, I need to find out how good a fit is possible with evenly spaced ampltiudes, and set the regularisation parameter so that the residual is on that order.  In fact, I already know how large the phase space expansion should be: the function being sampled should have a norm around 1.

The Picard condition is very obviously satisfied by the projection operator $|Γ〉$.  The left singular vectors can be any orthonormal basis of ket space; the singular values are all 1, so they don't reduce at all, and any ket has a preimage with the same norm as itself.  Functions in the null space of $|Γ〉$ don't have components in the left singular directions, so, if the problem is solved by expanding over singular vectors, they will not be present in the solution. 

%\topinsert\XeTeXpicfile catpicard.pdf width \hsize\vskip 5mm\it
%Picard plot for cat expansion. \endinsert

The next Figure is a Picard plot for the Galerkin discretisation $〈A|A〉\vec f=〈A|ψ〉$, where $\hat f$ is a vector of weights such that $|A〉\vec f$ approximates $|ψ〉$.  The singular value decomposition is $|A〉=|U〉SV$.  For the kets to be orthogonal, $|u_r〉$ must have support close to $n=r$.  The components of the cat have coherent amplitude 2, so their largest number component is $n=4$, and the components drop quickly as $n$ increases further.  Above $r=15$, the singular ket $|u_r〉$ has very small components on these number states, so it hardly overlaps $|ψ〉$.  At the same time, the singular kets start to have particle numbers too large to be consistent with the sample amplitudes $α_r$, so the singular values decrease.  This is shown in the plot of Fock state components of the singular kets.  In fact, they decrease more rapidly than the components of $|ψ〉$, and the discrete Picard condition is not satisfied: the discrete doesn't have a finite solution.  Seeing that the continuous operator $\G$ is perfectly conditioned, this is a very bad discretisation.

The final graph is an L curve when the discrete problem is Tychonov regularised.  It shows the expected shape, with minimum residual around $2×10^{-8}$.

%\topinsert\XeTeXpicfile catsws.pdf width \hsize\endinsert

%\topinsert\XeTeXpicfile catl.pdf width \hsize\endinsert

A common observation when states are expanded over an irregular grid is that, when two sample amplitudes are close together, the weights tend to be large, and cancel out.  This is not at all surprising when one considers that the coherent states $|α〉$ and $|α+h〉$ become nearly parallel in Hilbert space as $h→0$.  The figure demonstrates why weights of nearly parallel (or antiparallel) vectors tend to cancel out, and the second figure shows how this looks in phase space.

%\topinsert \XeTeXpicfile geometry-1.pdf
A 2 dimensional section through Hilbert space, where fixed vector $|Ψ〉$ is expanded as $|Ψ〉=a|α〉+b|β〉$.  As $|β〉$ becomes more parallel with $|α〉$, the components grow and cancel.  This occurs for all vectors outside the shaded segments, which shrink to nothing when $|α〉$ becomes parallel to $|β〉$.
%\endinsert

%\topinsert \XeTeXpicfile cancel.pdf width \hsize
The vacuum state was expanded over two coherent states, with amplitudes $±h$.  The amplitudes and the entire expansion of the least-squares fit are plotted in the first row of images.  The norm of the expansion vector, and the residual of the expansion, are graphed above as black lines.  As the amplitudes converge, the residual decreases—the leftmost phase space plot is very nearly exact—while the norm of the expansion stays around 1.  This coincides with the amplitudes passing through bright areas of the images, where the entire expansion has support.

The second row of images, and the red lines on the plot, show similar expansions of the normalised state parallel to $|0.1〉-|-0.1〉$.  This is almost exactly parallel to the derivative of the vacuum state with respect to the real part of its amplitude.  Again, the leftmost image is nearly an exact expansion.  This time, the entire expansion does not have support around $α=0$.  As the expansion amplitudes converge, the residual decreases, but the norm of the expansion vector explodes.  As explained in the text, the distance between two small amplitudes in phase space is very nearly the angle between the ket vectors for the corresponding coherent states.  Thus this is exactly the expansion depicted in the last figure.

If the two sample amplitudes were part of an irregular expansion, and there happened to be no other sample amplitudes nearby, a small component along the direction of the second state would dominate the expansion coefficients at these points.  But, given those sample amplitudes, that would be the only way to represent such a component.
%\endinsert

\beginsection{Adapting the sample amplitudes to preserve condition}

I'm getting a picture of how the final algorithm will look.  At each timestep, the directions belonging to large singular values of $|Dψ〉$ will be used for variational dynamics.  This can be done by solving at Tychonov conditioned problem, for example.  At the same time, the directions with small singular values will be used to keep the discretisation stable.

To solve our differential equation, we need to solve an iterated least squares problem.  We need to do so in a way that preserves the condition of $|A(z)〉$.  When we shift $z$ in the direction $\dot z$, how does the condition change?  The issue is the small singular values, so it makes sense to constrain the smallest singular value $σ_R$.  This is a function of $z$.  We proceed to calculate its derivative by first order pertubation theory.

Let 
$$|A(α)〉=\pmatrix{|α₁〉& \cdots &|α_R〉}
	=\pmatrix{\cdots &e^{-½|α_r|²+α_ra†}&\cdots}|0〉.$$
This is not an analytic function, so there is no Cauchy derivative such that $|A(z+h)〉=|A(z)〉+|DA(z)〉h+O(h²)$.  There are Wirtinger derivatives such that $|A(z+h)〉=|A(z)〉+|DA(z)〉h+|CA(z)〉h†$.  However, in this case it is easier to work with real and imaginary parts, defining
$$|A'(x,y)〉=\pmatrix{|x₁+iy₁〉&\cdots &|x_R+iy_R〉}
	=\pmatrix{\cdots &e^{-½x_r²-½y_r²+(x_r+iy_r)a†}&\cdots}|0〉.$$
The partial derivatives are easily determined to be
$$|∂_rA'(x,y)〉=\cases{
	\pmatrix{0 & (-x_r+a†)|x_r+iy_r〉 & 0} & $1≤ r≤ R$, \cr
	\pmatrix{0 & (-y_r+ia†)|x_r+iy_r〉 & 0} & $R<r≤ 2R$.}$$
Because $|A〉$ matrix-valued, its total derivative is some kind of third rank tensor, and its partial derivatives are simpler to work with.  The total derivative $Dσ_R$ is just a row vector, and will be used often.

Let the singular value decomposition of $|A(α)〉$ be $|A(α)〉=|U〉SV†$.  The standard ways to compute singular values use the matrix
$$M(x,y)=\pmatrix{0 & |A'(x,y)〉\cr 〈A'(x,y)| & 0}.$$
Here, $|A(x,y)〉$ maps weight vectors in $C^R$ to kets, while $〈A(x,y)|=VS〈U|$ maps kets back to column vectors of inner products.  The operation of the zeros is obvious; formally, they need to be a row of null kets and a column of null bras.  The matrix $M(x,y)$ operates on a vector comprising a ket above $R$ complex numbers.  Now, the vector
$$x={1\over \sqrt2}\pmatrix{|u_R〉 \cr v_R}$$
is a normalised eigenvector  of $M$, belonging to the eigenvalue $σ_R(x,y)$.  As is well known in first-order pertubation theory, the derivative of this eigenvalue with respect to $x$ and $y$ is given by
$$∂_rσ_R=x†(∂_rM)x
	=½\pmatrix{〈u_R| & v†_R}\pmatrix{0 & |∂_rA'(x,y)〉\cr 〈∂_rA'(x,y)| & 0}\pmatrix{|u_R〉 \cr v_R}=\Re\left(〈u_R|∂_rA'〉v_R\right).$$
Substituting the partial derivatives gives
$$∂_rσ_R=\cases{\Re\left(〈u_R|(-x_r+a†)|α_r〉v_R^r\right) &  $1≤ r≤ R$, \cr
	\Re\left(〈u_R|(-y_r+ia†)|α_r〉v_R^r\right) & $R<r≤ 2R$.}$$

The accuracy of the pertubation method is shown below.  A vector of 5 amplitudes was drawn from a normal distribution around amplitude 2, shown in the upper plot.  The lower plots show how the smallest singular value of $|A〉$ varies as the amplitudes shift.  The five columns show changes in the five amplitudes.  The black lines show how $σ_R(x,y)$ varies with the real part $x$, the red lines how it varies with $y$.  The dotted lines show the linear approximation derived above by pertubation theory, and the second row of plots show the error in the approximation.  The linear approximation is good; presumably it holds for changes small compared to the separation of the amplitudes.

%\topinsert 
%\vskip -40mm
%\XeTeXpicfile psfa.pdf width \hsize
%\vskip -90mm
%\XeTeXpicfile psv.pdf width \hsize\endinsert

Given the gradient $Dσ_R$, we wish to find a direction where $σ_R$ increases most rapidly for a given change in $|ψ〉$.  Given a small change $|dψ〉$ lying in the tangent plane, the corresponding change in $z$ is given by $dz=〈Dψ^{+}|dψ〉$, where $〈Dψ^{+}|$ is the pseudoinverse of $|Dψ〉$.  The change in $σ_R$ is $dσ_R=Dσ_Rdz=Dσ_R〈Dψ^{+}|dψ〉$.  So the gradient direction in Hilbert space is $|Dψ^{+}〉Dσ_R†$; the corresponding direction in $z$ space is $〈Dψ^{+}|Dψ^{+}〉Dσ_R†$.  In terms of the SVD $|Dψ〉=|U〉SV†$, this is $〈U|S^{-2}|U〉Dσ_R†$.  Here, $|ψ〉$ is considered a function of $4R$ real variables instead of $2R$ complex ones.

There are now three goals for optimisation of $\dot z$:
\item{1.} The ket $|ψ(z)〉$ solves Schrödinger's equation.  I.e., $|Dψ(z)〉\dot z=H|ψ(z)〉$.
\item{2.} Singular vectors of $|Dψ〉$ belonging to small singular values do not dominate the solution.  This is controlled by a Tychonov condition, $\dot z=0$, or perhaps by truncation.
\item{3.} The condition of $|A(z)〉$ improves.  This can be set up as a least squares condition $Dσ_R(z)\dot z+Cσ_R(z)\dot z*=|σ_R||\dot z|_{\rm opt}$, given a desired rate of change of $z$.  No doubt there is a better way.
\item{4.} There might need to be fourth condition, to prevent amplitudes with small weights from wandering off to infinity.

Putting these conditions together, with regularisation parameters $λ$ and $ε$, gives the least squares problem
$$\pmatrix{|Dψ(z)〉& 0 \cr λI & 0 \cr Dσ_R(z) & Cσ_R(z)}
	\pmatrix{\dot z\cr \dot z*} ≈
	\pmatrix{H|ψ(z)〉\cr 0\cr |σ_R||\dot z|_{\rm opt}}.$$
This raises a technical problem, because we need to constrain $\dot z*$ to be the conjugate of $\dot z$.  In the space of complex numbers, this constraint is not linear, because $(cz)*≠cz*$.  However, that is true if $c$.  Moreover, if $z=x+iy$, then $|z-w|²=|x-\Re w|²+|y-\Im w|²$, so a least squares problem in $z$ has the same solution as the equivalent least squares problem in $x$ and $y$.

Consider the problem 
$$ \pmatrix{A & B}\pmatrix{z\cr z*}≈c,$$
with $z∈C^m$, $c∈C^n$, and $A,B∈C^{m×n}$.  If $z=x+iy$, the left hand side can be rearranged
$$\pmatrix{A & B}\pmatrix{z\cr z*}
	=\pmatrix{A & B}\pmatrix{I&iI\cr I&-iI}\pmatrix{x\cr y}
	=\pmatrix{A+B & iA-iB}\pmatrix{x\cr y}.$$
Separating the rows into real and imaginary parts gives
$$\pmatrix{\Re (A+B) & \Re(iA-iB)\cr\Im(A+B) & \Im(iA-iB)}\pmatrix{x\cr y}
	=\pmatrix{\Re A+\Re B & \Im B-\Im A\cr\Im A+\Im B & \Re A-\Re B)}
		\pmatrix{x\cr y}≈\pmatrix{\Re c\cr \Im c}.$$

Below is a plot of the condition number of $|Dψ(z)〉$ for a superposition, as the value of $z$ varies by $0.1$ in every direction.  The top line shows the effects of changing $φ$, the bottom line $α$, components are shown left to right.  Changes to real parts are black, imaginary parts red.  There are a few things to note.  The condition is very bad, with number $1.4×10^{11}$.  This is entirely due to the similar amplitudes of components 1 and 3: those graphs are on a different scale to the others, and it's logarithmic.  As they move closer and become identical, the condition number increases $10^{14}$.  No doubt it would go higher if we sampled the graphs more finely, but this is huge—reciprocal floating point epsilon is $5×10^{15}$.  If the separation between these components is doubled, the condition number improves by an order of magnitude.

%\centerline{\XeTeXpicfile j1.pdf width \hsize}

%\centerline{\XeTeXpicfile j2.pdf width \hsize}

The top line indicates that the condition number is independent of the phases of the components.  I didn't forsee that, but it isn't too surprising: we can change the coefficients of the superposition in any direction we like, regardless of their current values, and the derivative is only sensitive to changes.

\beginsection{Ways the problem has been simplified for this paper}

The problem we have discussed so far is to expand a state $|ψ〉$ over a set of coherent states, in the form $|ψ〉=|A〉d$.  In order for a this state to be represented on a computer, the coherent amplitudes $A$ and the coefficients $d$ would all have to be stored in memory.  In a simulation, we will have an expansion $d$ for $|ψ〉$, which we want to perturb to represent $|ψ〉+|δψ〉$.  The expansion operator $|A〉$ is linear, so we could expand $|δψ〉=|A〉δd$ and add $δd$ to $d$  However, if the simulation is to represent a many-mode state efficiently, the amplitudes will likely have to vary over time as well as the coefficients.  Also, it might sometimes be more efficient to vary the ampltiudes.  For example, if a coherent state was rotating in an oscillator, it might be simpler to rotate the $α_j$ than to vary the weights.

We then have a variational problem $|\dot ψ〉=|Dψ(z)〉\dot z$, where
$$z=\pmatrix{A\cr d},$$
and the expansion operator $|Dψ(z)〉$ includes the derivatives of the coherent states with respect to their amplitudes (notation?).

These two problems are closely related.  In fact,
$$|Dψ(z)〉=
	\lim_{h→0}\pmatrix{|A+½h〉&|A-½h〉}
	\pmatrix{{I\over 2}&{Id\over h}\cr{I\over 2}&-{Id\over h}}.
$$
The variational problem is equivalent to expanding the state over a frame of $2R$ coherent states, clustered in tight pairs, then taking sums and differences of the expansion coefficients.  In fact, the vector $d$ could be absorbed into the shifts $+½h$ and $-½h$, so that the transformation matrix was even simpler.  This offers other approaches to regularisation: we could expand with regularised coefficients, and apply the transformation to the result.

When two amplitudes $α_r$ and $α_s$ are close together, three linearly independent $d$ cause $|Dψ(z)〉d$ become parallel.  These being $|∂_rψ〉- |∂_sψ〉$, $|∂_{R+r}ψ〉$, and $|∂_{R+s}ψ〉$.  Two nearby points effectively become one, as far as the variational problem is concerned!

Since we have a totally stable solution to the problem, why would we try anything else?  Because we're iterating the problem, and we need to preserve our ability to solve it at future steps.  If the amplitudes form a grid, which covers all the energies the system could possibly access, we're set: we'll always be able to expand over them.  However, we can't do that for many mode systems: there will always be accessible states of the system whose support in phase space is remote from the current grid.  The set $|Dψ〉$ will have some small singular values, corresponding to directions where the amplitudes and samples are adjusted in a way that preserves the wavefunction.  We are free to add these as we like, in order to maintain a grid on which we can expand $|ψ〉$ in the future.

%% Fock space is cheating

Expanding the quartic oscillator over Fock states allows it to be solved very stably, and as precisely as desired by rotating the phases of the expansion coefficients.  Also, it is easy to find the expansion over Fock states of a superpositon of coherent states.  This means that the stability of time discretisation can be removed from the problem entirely.  The state can be propagated exactly, and the coherent states adjusted to fit.  The most direct way to do so is by nonlinear optimisation at every timestep; Matlab provides the necessary routines in a convenient package.  Unfortunately, this is not at all stable.  

In a sense, expanding over Fock states is cheating.  If it were practical to do this, there would be no point using a variational method: these would be employed in many-particle problems, where an orthonormal basis such as the Fock states would be unmanagably large.  In principle, the operations required for the second and third methods could be performed on the superpositions of coherent states, using the normal equations, without projection on an orthonormal basis; in a many-particle problem, they would have to be.  However, this would raise the question whether those least-square solutions from the normal equations were stable.  By using an orthonormal basis, we can be more confident of this.

To do this, we need some bracket matrices.  We have
$$|Dψ(φ,α)〉=(1,a†)|ψ(φ,α)〉.$$
The brackets with the number states are
$$〈m|H/\hbar|n〉=n(n-1)δ_{mn}$$
and
$$〈n|Dψ(φ,α)〉=e^φ\pmatrix{{α^n\over \sqrt{n!}} &  α^{n-1}\sqrt{n\over (n-1)!}},$$
as is the bracket of the quartic oscillator Hamiltonian $a^{2\dagger}a²$
$$〈n|H/\hbar|ψ(φ,α)〉=e^φα^n\sqrt{n(n-1)\over(n-2)!}.$$


The instability in the dynamics is caused by the near-singularity of $|Dψ〉$.  If a singular value decomposition $|Dψ〉=|U〉SV†$ is taken, the singularity is indicated by small singular values on the diagonal of $S$.  It is often measured by the condition number of $|Dψ〉$, the ratio of the largest to the smallest singular value.  Let $C(z)$ be the condition number of $|Dψ(z)〉$.  As part of the regularisation process, it would be useful to avoid changing $z$ in directions that increase $C(z)$.  For example, if two coherent states had real amplitudes, and the real part of the expected amplitude needed to increase, it would be done by shifting the right hand one away from the left hand one, not the left hand one closer to the right hand.  Is there a way to make a computer do that?

The least squares problem can be solved using the singular value decomposition: to approximate a state $|h〉≈|Dψ(z)〉dz$, we set $dz=VS^{-1}〈U|h〉$.  The first problem is that some of the $σ_i$ on the diagonal of $S$ will be small, because the $|u_i〉$ that correspond to them change different components of the superposition $|ψ(z)〉$ in ways that nearly cancel out.  So a error in the discrete representation of $|h〉$ in the direction $|u_i〉$ will cause a very large error in the least squares $dz$ in the direction $v_i$.   The way to avoid such large errors is to shift the small singular values of $S$ up to a lower bound $ε$, and thus reduce the large singular values of $S^{-1}$ below an upper bound $ε^{-1}$.  There are many and varied ways of doing that. 

\beginsection{Regularising the inital fit equation}

As discussed below, the time-dependent Schrödinger equation for a quartic oscillator is numerically difficult, no matter how it is discretised.  The following problem aimed to solve an intrinsically simple differential equation, 
$$ {d\over ds}|ψ(z(s))〉=|ψ₀〉-|ψ(z(s))〉.$$
Here, $|ψ₀〉$ was the half cycle cat state; $|ψ(z)〉$ had 15 components, and was initialised by sampling random, and linearly fitting weights.  The result is shown on the left below.  The clusters of complementary colours indicate components that nearly cancel each other out; these should not be present when the condition of $|A〉$ is constrained, but are expected from random sampling.  They might also indicate too small a Tychonov parameter: large parameters will tend to give nearby points equal weights.

%\topinsert \XeTeXpicfile expens.pdf width \hsize \endinsert

The equation was solved, in the script {\tt hzero.m}.  The results are shown below, and the final superposition is plotted on the right above.  The second graph below isthe smallest singular value of $|A〉$.  Many draws of initial samples give the exponential decay one would expect.  This one gave two surprises: initially, the distance between $|ψ〉$ and $|ψ₀〉$ increases with time.  This is entirely wrong, according to the differential equation.  The initial state was a good approximation to $|ψ₀〉$, so there might not be much scope to improve the approximant.  However, it should certainly not move away from the target.

The variational method was the least distance in Hilbert space, with Tychonov regularisation.  Time was discretised by Euler's formula, which should solve exponential decay very stably.

The second surprise is at large $s$, where the approximant improves, but so does the condition of $|A〉$.  In most draws of initial samples, the approximant improves by clustering the amplitudes towards the states $|2i〉$ and $|-2i〉$, which makes the conditioning worse.

%\topinsert \XeTeXpicfile exp.pdf width \hsize \endinsert

There are a few things to note about the initial trajectory.  The initial weights already satisfy a least squares condition.  Therefore, the intial change in the superposition should be entirely in the amplitudes: the weights are already optimal.  If that isn't the case, the regularisation of the expansion problem would have to be inconsistent with that of the variational problem, so that we're accepting variational changes that we previously rejected when we found the initial state.

\beginsection{Quantum dynamics in Hilbert space}

If Hilbert space is considered as a real vector space, complex multiples of the eigenstates of the Hamiltonian form planes instead of lines.  The eigenstates evolve as $e^{iωt}$, so, when the dynamics are projected onto an eigenplane, the state traces out a circle, with radius equal to the absolute value of the the coefficient of the eigenstate when the initial state is expanded.  Different eigenstates circle the origin with different frequencies.

The semi-implicit formula has a special property: when the trajectory of $u(t)$ is a circle, the $vⁿ$ lie exactly on that circle, and the discretisation error is equivalent to rescaling time by a factor that depends on the frequency and the discrete timestep.  In Hilbert space, this means that the projections of the state onto eigenplanes still circle the origin exactly: only the frequencies are altered by discretisation.  So the effect of discretising quantum dynamics using the semi-implicit formula is merely to shift the energies of the eigenstates.  Of course, this can be a significant effect: it's the only difference between the harmonic and quartic oscillators.

%% Discrete time and stiffness

I used three discretisation formulae for ODE, with different regions of stability.  As a first step, the Hamiltonian and a coherent state with real amplitude 2 was expanded over Fock states $|0〉$ through $|10〉$, and the resulting equations for the components solved directly.  The Hamiltonian is
$$H=\hbar a^{2\dagger}a²,$$
so the Fock states are eigenstates, although the spacing of their energies is uneven.  The quantum state at time $t$ was expanded over a truncated Fock basis $|N〉=\pmatrix{|1〉 & |2〉 & ⋯ & |N〉}$, with a column vector of coefficients $c(t)$, as
$$|ν(c(t))〉=|N〉c(t).$$
Schrödinger's equation expands to
$$|N〉Dc(t)=|D(ν\circ c)(t)〉={H\over i\hbar}|ν(c(t))〉=|N〉〈N|{H\over i\hbar}|N〉c(t),$$
whence
$$Dc(t)=-i〈N|{H\over \hbar}|N〉c(t).$$
Usually, these expansions would be approximate due to basis truncation.  In this case, the Fock states diagonalise $H$, so they are exact.  However, in general the initial state can only be approximately expanded over $|N〉$.  Time was discretised with the explicit Euler formula, which replaces the differential equation
$$Du(t)=f(u(t))$$
with a difference equation
$$v^{n+1}=vⁿ+τfⁿ,$$
where $tⁿ=nτ$, $vⁿ$ is meant to approximate $u(tⁿ)$, and $fⁿ=f(vⁿ)$.  Let
$$vⁿ=cⁿ\qquad{\rm and}\qquad fⁿ=-i〈N|{H\over\hbar}|N〉c(t).$$
Since $〈N|H|N〉$ is diagonal, the matrix product can be performed sparely as a dot product.  The results are shown below.  The Fock states are eigenstates of the Hamiltonian, so their amplitudes should be constant: in the exact solution of Schrödinger's equation, all the ratios plotted are exactly 1.

%\centerline{\XeTeXpicfile dry.pdf width \hsize}

These results are the textbook signs of stiffness.  As the time step is reduced from values on the order of 1, the discretised solutions diverge hugely, before converging at very small time steps.  In the standard theory of discrete approximations to $Du(t)=f(u(t))$, stability is determined by the eigenvalues of the linear transform $Df$.  In Schrödinger's equation, $f$ is the Hamiltonian, a linear transform, so in our expansion, $df=f=-i〈N|H/\hbar|N〉$, and the relevant eigenvalues are $-iEⁿ/\hbar=-in(n-1)$.  The stiffness of a differential equation is often measured by the ratio of the sizes of these eigenvalues.  For the quartic oscillator, some eigenvalues are zero, so this is not a sensible measure.  However, the largest eigenvalue, which often determines the timestep, increases quadratically with the size of the Fock basis as $N(N-1)$.

The stability region of the explicit Euler formula is the disk $|\bar k-(-1)|≤1|$.  So it is unstable for any Jacobian with complex eigenvalues, which is certainly the case for a Hamiltonian.  The implicit Euler formula, 
$$v^{n+1}=vⁿ+τf^{n+1},$$
has the opposite stability region.  For the oscillator, this becomes
$$c^{n+1}=cⁿ-iτ〈N|{H\over\hbar}|N〉c^{n+1},$$
so $c^{n+1}$ satisfies the linear equation
$$(1+iτ〈N|{H\over\hbar}|N〉)c^{n+1}=cⁿ.$$
This is still diagonal for the quartic oscillator.  To order $τ$, the implicitn formula is the same as the explicit one.  Obviously the higher powers of $τ$ make the difference when $τ$ is finite.

The results of the backward Euler formula are the opposite of the Euler formula.  Instead of increasing, the amplitudes of the larger number states decrease nearly to zero.  This distorts the trajectory of $〈a(t)〉$, as show below.

A third method is the semi-implicit one, with the formulae
$$v^{\prime n}=vⁿ+½τf^{\prime n}\qquad v^{n+1}=vⁿ+τf^{\prime n}.$$
With the Hamiltonian, this becomes
$$c^{n+1}=cⁿ-iτH\biggl(1+{iτ\over 2}H\biggr)^{-1}cⁿ.$$

The solutions from method 3 provide a test of the stiffness of the ODEs defined in parameter space by the least squares problem.  The step in Hilbert space is very nearly the same from one time step to the next.  If the step in parameter space changes dramatically, this would suggest that the variational solution is very sensitive to the state, and thus to some of the parameters.  This would indicate that the ODEs are stiff.

\beginsection{Quantum dynamics in phase space}

The following graphs show a coherent state, with real amplitude 2, propagating in a quartic oscillator potential.  The propagation was done exactly, in a Fock space truncated at $|19〉$; at times $π/10, π/5, 3π/10,\ldots,π$, a 10 component superposition was fit to the state by the routine {\tt cohfit5}.  The SVD of $|Dψ〉$ was computed for each of these superpositions, and the Picard condition investigated.  It appears that this discretisation is inherently ill-posed.

%\topinsert \XeTeXpicfile spsns.pdf width \hsize
%Amplitudes of components \endinsert

%\topinsert \XeTeXpicfile traj.pdf width \hsize
%Left: expectation value $〈a〉$ at sampled times.  Right: residual of fit by superposition (solid) and condition number of Jacobian of fitted superposition (dotted) \endinsert

%\topinsert \XeTeXpicfile picard.pdf width \hsize
%Picard plots at fitted times.  Circles are singular values, crosses components of LSVs, red stars components of RSVs.  The discrete Picard condition is clearly not satisfied. \endinsert

%\topinsert \XeTeXpicfile svnum.pdf width \hsize
%Expected particle number for left singular vectors.  The svs with small sws have large particle number.  When $H$ is quartic, these have large components in $H|ψ〉$.  In order for the svs to be orthogonal, the points are likely to lie around the line ${\rm sv}=n$. \endinsert

In the second choice, the state is approximated as a superposition of coherent states,
$$|ψ(z(t))〉=∑_{φ,α∈z(t)}e^{φ+αa†}|0〉=∑_{φ,α∈z(t)}e^{φ+½|α|²}|α〉.$$
Time is discretised immediately, setting
$$vⁿ=|ψ(zⁿ)〉\qquad{\rm and}\qquad fⁿ={H\over i\hbar}|ψ(zⁿ)〉,$$
so that the midpoint formula becomes
$$|ψ(z^{n+1})〉≈|ψ(z^{n-1})〉+{2τH\over i\hbar}|ψ(zⁿ)〉.$$
The approximation is due to basis truncation, as always in a least squares sense.  Writing the unknown $|ψ(z^{n+1})〉$ in terms of $|Dψ(zⁿ)〉$ and expanding over a Fock basis gives
$$ 〈N|Dψ(zⁿ)〉\bigl(z^{n+1}-zⁿ\bigr)≈〈N|ψ(z^{n-1})〉-〈N|ψ(zⁿ)〉-2iτ〈N|H|ψ(zⁿ)〉.$$
This matrix equation can be solved for $z^{n+1}-zⁿ$, which can be used to update $z$.  

The third choice, the conventional variational one, is to discretise the state first, setting
$$vⁿ=zⁿ\qquad{\rm and}\qquad |Dψ(zⁿ)〉fⁿ≈{H\over i\hbar}|ψ(zⁿ)〉.$$
The midpoint formula is used exactly, and $fⁿ$ satisfies the least squares problem
$$〈N|Dψ(zⁿ)〉fⁿ≈-i〈N|Dψ(zⁿ)〉{H\over \hbar}|ψ(zⁿ)〉.$$
Usually, this would be expanded over $〈Dψ(zⁿ)|$ to give a set of normal equations.  However, when $|Dψ(zⁿ)〉$ is near singular, the numerical stability of the normal equations is suspect.  Expanding over an orthonormal basis removes this confounding instability.

%\topinsert \XeTeXpicfile flwc0n10.pdf width \hsize \endinsert

%\topinsert \XeTeXpicfile flwc0n5.pdf width \hsize \endinsert

The above plots show my first attempt at this.  The initial state was coherent, with real amplitude 2.  A superposition of 5 coherent states was initialised with amplitudes normally distributed around 2, the norm of the superposition being 1, and the components having equal norms.  This superposition was fit to the coherent state by minimising their distance in Hilbert space; for stability, the weights were constrained so that no component had a norm less than 0.1.  This was done by the Matlab routine {\tt fmincon}.  The initial state was truncated to Fock states $|0〉$ to $|10〉$, and propagated exactly in a Hamiltonian $\hbar a^{2\dagger}a²$ with timestep 0.03.  After each step, the fitting of the superposition to the exact state was repeated.

The results are shown in the first Figure.  The left hand plot shows the complex amplitude of the exact solution in black, and the fitted solution in blue.  The right hand plot shows the residual of the fit as a solid line, and the condition number of $|Dψ(z)〉$ at the fitted $z$ as a dotted line.  The program stopped after a short time, when the fitted amplitudes caused overflow when expanded over Fock coefficients.  

The second Figure shows the same simulation, with the states expanded over Fock states $|0〉$ to $|5〉$.  This has a significant effect on the truncated exact solution, adding an extra loop to the amplitude.  The fitting works a bit longer, but still fails at a very short time.

%\centerline{\XeTeXpicfile flwc2n10.pdf width \hsize}

%\centerline{\XeTeXpicfile flwc2n5.pdf width \hsize}

The next two plots show a different fitting strategy.  Instead of constraining the weights, a cost term was added in quadrature to the Hilbert space distance, proportional to the logarithm of the condition number of $|Dψ(z)〉$ for the fitted $z$.  The constant of proportionality was $10^{-5}$; this was adjusted until a good fit was obtained.  The most apparent thing is that the fit is much more stable: this is being done with timestep 0.1 instead of 0.3, but the fit stays stable for the whole period.  (I had to redraw the initial amplitudes a few times before that happened.)  It isn't a very good fit through the middle of the period: the residual increases to order 1.  The really interesting thing is what happened when, by accident, I truncated the Fock basis at $N=5$.  As before, this perturbs the trajectory in Hilbert space.  However, the perturbed trajectory can be fit very well: the residual is between $10^{-2}$ and $10^{-5}$, and this is done with quite well conditioned superpositions.

So even without stiffness in a discrete time propagation, there is something in the high energy components of $H|ψ〉$ that makes fitting the state with coherent states unstable.  This isn't too surprising: these high energy components are rapidly oscillating, and will by magnified when the inverse problem is solved with fairly smooth, low amplitude coherent states.  To fit the oscillating components, we need coherent states with large amplitudes, and low weights.  But the low weights will increase the condition number of $|Dψ〉$, so the regularisation scheme will avoid these, unless there are many components in the superposition and all their weights are low.  Let's investigate what happens as the number of components and the Fock truncation are varied together.

The things I've noticed so far are:
When a superposition is iteratively fitted to the exactly propagated state, adding a term to the cost proportional to the Jacobian $|Dψ〉$ keeps the fit stable for an entire cycle.  It isn't accurate, though.

Measuring state differences in a truncated fock space makes the fit stable and accurate.  Of course, the truncated state is different from the original one, so the original state is not fit precisely.  When this was done, the least squares problem involved 6 fock coefficients, and the amplitude vector has 10 elements.  So almost all the truncated problems were underdetermined: some singular values of the Jacobian are exactly zero.  Matlab claims that backslash solves underdetermined problems by giving a solution with as many zero elements as possible.  The documentation doesn't say how it chooses which elements to make zero.  That's different from the other regularisation methods we've tried.  Of course, the problems are being solved inside Matlab's optimisation routine, and I don't know what it's doing.

%% The complete algorithm

The final algorithm would propagate $|ψ〉$ in Hilbert space by a semi-implicit method, solving a variational problem to represent each state.  This is a slight generalisation, because you need to project $H|ψ_{n+½}〉$ in to the span of $|Dψ_n〉$, and the ensembles at the two times are slightly different.  It might help to propagate from $n$ to $n+½$ by changing only weights, leaving the same ensemble.  On the other hand, the change is only that the $α*$ become $β*$, so it should be managable.

There shouldn't be an need for fancy methods to fit the initial state: one can take a guess $z₀$, and variationally solve the equation $|Dψ(z)〉dz=|ψ₀〉-|ψ(z)〉$.  One could even change the conditioning rules as the state propagates, so that $|Dψ〉$ is aligned with $|ψ₀〉-|ψ(z)〉$ to start with, then with $H|ψ(z)〉$ later on.  That way, you start the actual simulation with a well conditioned approximant.

%% Numerical experiments

I want to measure the numerical stability of the variational dynamics derived from Schrödinger's equation.  I can do that by measuring how the discretisation error converges as the timestep is reduced.  The second hypothesis is that the instability is due to stiffness, which can be tested by discretising time in ways with a variety of stability regions, and seeing whether the stability differs.


%%%%

When the least squares problem is solved with a different distance function, which sets distances along the large $N$ directions to zero, the fit fails on the first step.  Note that this isn't really a distance.  Also, I did it by solving $P|h〉=P|Dψ〉dz$; when $P$ truncates the Fock space, the operator $P|Dψ〉$ is singular.  Doing a weighted least squares fit, with the many particle directions having low weights, would be near-singular.


\bye