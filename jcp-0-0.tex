\documentclass[aip,jcp,graphicx,draft]{revtex4-1}
%\documentclass[aip,reprint]{revtex4-1}

\begin{document}

% Use the \preprint command to place your local institutional report number 
% on the title page in preprint mode.
% Multiple \preprint commands are allowed.
%\preprint{}

\title{On the numerical stability of static Gaussian wave packets} %Title of paper

% repeat the \author .. \affiliation  etc. as needed
% \email, \thanks, \homepage, \altaffiliation all apply to the current author.
% Explanatory text should go in the []'s, 
% actual e-mail address or url should go in the {}'s for \email and \homepage.
% Please use the appropriate macro for the type of information

% \affiliation command applies to all authors since the last \affiliation command. 
% The \affiliation command should follow the other information.

\author{Rodney E. S. Polkinghorne}
\email[]{rpolkinghorne@swin.edu.au}
%\homepage[]{Your web page}
%\thanks{}
%\altaffiliation{}
\affiliation{Centre for Quantum and Optical Science, Swinburne University of Technology}

% Collaboration name, if desired (requires use of superscriptaddress option in \documentclass). 
% \noaffiliation is required (may also be used with the \author command).
%\collaboration{}
%\noaffiliation

\date{\today}

\begin{abstract}
The ability to expand high-energy states over a set of non-orthogonal wave packets causes the wave packet equations of motion to become stiff.  On the other hand, a dense set of wave packets is required to have stable expansions of wave functions.  Tychonov regularisation of the expansion operator suppresses the energies of the high-energy states, while retaining all the physics that happens slower than a given time scale.
\end{abstract}

\pacs{}% insert suggested PACS numbers in braces on next line

\maketitle %\maketitle must follow title, authors, abstract and \pacs

% Body of paper goes here. Use proper sectioning commands. 
% References should be done using the \cite, \ref, and \label commands
\section{Introduction}

Since Heller first proposed the method in 1976\cite{jcp-64-63}, there have been several attempts to compute the dynamics of quantum systems by use of coupled Gaussian wave packets.

Schemes such as \cite{jcp-132-244111,jcp-136-014109} have used moving grids of coherent states, to which all these considerations would apply.

\section{Ad-hoc regularisation}

{\tt Systems treated}

{\tt Where the method worked}

{\tt When instabilities were noticed}

{\tt What people tried to do about them}

{\tt Proposal for quantum fields}

{\tt Wavelets and Gabor frames in the 1980s}

\section{Differential equation formulae and stiffness}

\section{Expansions over wave packets}

{\tt Many wave packets, large number states can be represented, discretised $H$ has large evs, discretised equations of motion are stiff.}

{\tt Why do the least norm expansions of number states have large components near the origin of phase space?}

\section{Local regularisation}

{\tt Effects on evs, stabilisation of dynamics.}

\section{Prospects}

{\tt Other things go wrong when amplitudes are varied, need to study further.}

%\label{}
\subsection{}
\subsubsection{}

% If in two-column mode, this environment will change to single-column format so that long equations can be displayed. 
% Use only when necessary.
%\begin{widetext}
%$$\mbox{put long equation here}$$
%\end{widetext}

% Figures should be put into the text as floats. 
% Use the graphics or graphicx packages (distributed with LaTeX2e).
% See the LaTeX Graphics Companion by Michel Goosens, Sebastian Rahtz, and Frank Mittelbach for examples. 
%
% Here is an example of the general form of a figure:
% Fill in the caption in the braces of the \caption{} command. 
% Put the label that you will use with \ref{} command in the braces of the \label{} command.
%
% \begin{figure}
% \includegraphics{}%
% \caption{\label{}}%
% \end{figure}

% Tables may be be put in the text as floats.
% Here is an example of the general form of a table:
% Fill in the caption in the braces of the \caption{} command. Put the label
% that you will use with \ref{} command in the braces of the \label{} command.
% Insert the column specifiers (l, r, c, d, etc.) in the empty braces of the
% \begin{tabular}{} command.
%
% \begin{table}
% \caption{\label{} }
% \begin{tabular}{}
% \end{tabular}
% \end{table}

% If you have acknowledgments, this puts in the proper section head.
%\begin{acknowledgments}
% Put your acknowledgments here.
%\end{acknowledgments}

% Create the reference section using BibTeX:
\bibliography{your-bib-file}

\end{document}
%
% ****** End of file aiptemplate.tex ******
