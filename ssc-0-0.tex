% SIAM Article Template
\documentclass[review]{siamart1116}

% Packages and macros go here
\usepackage{lipsum}
\usepackage{amsfonts}
\usepackage{graphicx}
\usepackage{epstopdf}
\usepackage{algorithmic}
\ifpdf
  \DeclareGraphicsExtensions{.eps,.pdf,.png,.jpg}
\else
  \DeclareGraphicsExtensions{.eps}
\fi

%strongly recommended
\numberwithin{theorem}{section}

% Declare title and authors, without \thanks
\newcommand{\TheTitle}{Stability of the iterative semi-implicit formula for quantum dynamics} 
\newcommand{\TheAuthors}{Rodney E. S. Polkinghorne}

% Sets running headers as well as PDF title and authors
\headers{\TheTitle}{\TheAuthors}

% Title. If the supplement option is on, then "Supplementary Material"
% is automatically inserted before the title.
\title{{\TheTitle}\thanks{Submitted to the editors DATE.
\funding{This work was funded by the Fog Research Institute under contract no.~FRI-454.}}}

% Authors: full names plus addresses.
\author{
  Rodney E. S. Polkinghorne\thanks{Swinburne University of Technology
    (\email{rpolkinghorne@swin.edu.au}, \url{http://www.imag.com/\string~ddoe/}).}
}

\usepackage{amsopn}
\DeclareMathOperator{\sinc}{sinc}


%%% Local Variables: 
%%% mode:latex
%%% TeX-master: "ex_article"
%%% End: 

% Optional PDF information
\ifpdf
\hypersetup{
  pdftitle={\TheTitle},
  pdfauthor={\TheAuthors}
}
\fi

% The next statement enables references to information in the
% supplement. See the xr-hyperref package for details.

\externaldocument{ex_supplement}

% FundRef data to be entered by SIAM
%<funding-group>
%<award-group>
%<funding-source>
%<named-content content-type="funder-name"> 
%</named-content> 
%<named-content content-type="funder-identifier"> 
%</named-content>
%</funding-source>
%<award-id> </award-id>
%</award-group>
%</funding-group>

\begin{document}

\maketitle

% REQUIRED
\begin{abstract}
The iterative semi-implicit formula, or some variant of it, is often
used to solve stochastic partial differential equations that arise
in quantum optics and in atom optics.  When this formula is applied
to a quantum system with unitary dynamics, stability depends on a
kind of Nyquist condition, where the angular frequency corresponding
to the largest eigenvalue of the discretised hamiltonian is at most
twice the reciprocal time step.  The formula develops parasitic
solutions very abruptly as this condition is exceeded, which makes
it troublesome for practical use, but quite useful as a test of the
stability of discretisation methods.
\end{abstract}

% REQUIRED
\begin{keywords}
  example, \LaTeX
\end{keywords}

% REQUIRED
\begin{AMS}
  68Q25, 68R10, 68U05
\end{AMS}

\section{Introduction}

% formula equations
The semi-implicit formula is a rule for integrating systems of
ordinary differential equations \cite{jkp-132-312,jkp-93-144}.  It
is most often used as a method of lines, to solve stochastic partial
differential equations that occur when phase space Monte-Carlo
methods are used in quantum optics and in atom optics \cite{pra-58-4824}.
These equations resemble Schr\"odinger's equation,
\begin{equation}\label{eq:vi}
    \frac{\partial\psi}{\partial t} = -i(-\nabla^2 + V)\psi,
\end{equation}
with additional nonlinear and random terms.  The spatial axes are
discretised, often using spectral derivative operators, to give a
system of ordinary or stochastic differential equations
\cite{2000-Trefethen-Spectral}.  These equations are integrated
by the semi-implicit formula.

% iterations
For a system of equations
\begin{equation}
	{ds\over dt}=f(s),
\end{equation}
the semi-implicit formula with time step~$h$ is
\begin{equation}\label{eq:i}
    s_{n+1}=s_n+hf_{n+\frac12},
	\qquad\text{where}\qquad 
    f_{n+\frac12}=f\left({\textstyle\frac{s_n+s_{n+1}}{2}}\right).
\end{equation}
It is not obvious how to implement this formula, because~$s_{n+1}$
and~$f_{n+\frac12}$ need to be evaluated self-consistently.  The
usual way is a simple iteration \cite{swx-5-12}.  To start, $s_{n+1}$~is initialised
to~$s_n$, and~$f_{n+\frac12}$ to~$f(s_n)$.  Then each iteration
evaluates Equation~\cref{eq:i}, and updates the
values of~$s_{n+1}$ and~$f_{n+\frac12}$.  More precisely, the
iterations generate a sequence of~$s_{n+1}^{(j)}$ and~$f_{n+\frac12}^{(j)}$,
with
\begin{subequations}
\begin{align}
	\label{eq:iv} s_{n+1}^{(0)}&=s_n,\qquad f_{n+\frac12}^{(0)}=f(s_n),\\
	\label{eq:v} s_{n+1}^{(j+1)}&=s_n+hf_{n+\frac12}^{(j)},
		\qquad f_{n+\frac12}^{(j+1)}=f\left({\textstyle\frac{s_n+s_{n+1}^{(j)}}{2}}\right)\qquad \text{when $j\ge0$}.
\end{align}
\end{subequations}
This continues until the sequences converge, which is usually
achieved after a small, fixed number of iterations.  Any fixed point
of the iteration satisfies
Equation~\cref{eq:i}.  The remaining question is under what
conditions the iterations converge.

% steps stability
Generally, when a partial differential equation is discretised by
a given formula, the time step needs to be reduced in proportion
to the grid spacing in order for the discretised solution to remain
stable \cite{2000-Trefethen-Spectral}.  No one has analysed the stability of the
semi-implicit formula with iterations.  The stable time step has
had to be determined by trial and error, a time consuming process
for three-dimensional wave functions that are being simulated on
large grids.

This paper analyses the semi-implicit formula, in the case that it
is used to integrate linear Hamiltonian dynamics as in
Equation~\cref{eq:iii}.  The main result is that the stability
depends on a Nyquist-like condition, and the iterations converge
if and only if the eigenvalues of the hamiltonian are bounded by the reciprocal time step~$2/h$.

\section{Stability analysis}

% basis coefficients
When the semi-implicit rule is applied to \cref{eq:vi}, or a
stochastic or nonlinear variant of it, the laplacian operator is usually
evaluated by a spectral method \cite{2000-Trefethen-Spectral}.  This is equivalent to expanding
the wave function~$\psi(x)$ over a basis of sinc wave packets
\begin{equation}
	\psi(x) = \sum_m s^m\phi_m(x),
\end{equation}
where, for a spatial grid of step~$l$,
\begin{equation}
	\phi_m(x) = \frac1{\sqrt{l}} \sinc\left(\textstyle\frac{x}{l} - m\right).
\end{equation}
This paper uses the convention
\begin{equation}
	\sinc x = \frac{\sin\pi x}{\pi x},
\end{equation}
under which the set of functions~$\phi_m$ is an orthonormal basis
for the space of functions limited to spatial bandwidth~$1/2l$.
The spectral method reduces Equation~\cref{eq:vi} to an ordinary
differential equation for the complex coefficient vector~$s$ of the
form
\begin{equation}\label{eq:iii}
	{ds\over dt}=-iHs,
\end{equation}
where~$H$ is a Hermitian matrix with elements
\begin{equation}
	H_{mn} = \int \phi_m(x) (-\nabla^2 + V(x)) \phi_n^\ast(x)\,dx.
\end{equation}

If the formula of Equation~\cref{eq:i} could be applied directly
to Equation~\cref{eq:iii}, the computed solution would be very
stable.  The coefficient vector~$s_n$ can be expanded over the
eigenvectors~$v_k$ of~$H$, as $s_n=\sum_k c^kv_k$.  \Cref{fig:iii}
shows the trajectory over a time step from~$s_n$ to~$s_{n+1}$,
projected onto the complex plane spanned by the eigenvector~$v_k$.
The exact solution that satisfies Equation~\cref{eq:iii}, with
initial condition~$s_n$, is~$s(h)=\sum_k c^ke^{-i\omega_kh}v_k$,
and its projection lies on the circle around the origin of
radius~$|c_k|$.  Here~$\omega_k$ is the eigenvalue of~$H$ corresponding to the eigenvector~$v_k$.

\begin{figure}[htbp]
  \centering
  \includegraphics{semii.1}
  \caption{A step of the semi-implicit formula, projected onto the
  complex plane spanned by the eigenvector~$v_k$ of~$H$.  The
  computed solution~$s_{n+1}$ lies on the same circle as the
  exact solution~$s(h)=\exp(-ihH)s_n$, but it rotates with angular frequency lower
  than~$\omega_k$.  See text for details.}
  \label{fig:iii}
\end{figure}

Suppose that the point~$s_{n+1}$ also lies on this circle.
Then the projection of~$f_{n+\frac12}=-\frac{i}2H(s_n+s_{n+1})$ is parallel to the
chord from~$s_n$ to~$s_{n+1}$.  The midpoint of the chord can be
brought towards and away from the origin by moving~$s_{n+1}$ around
the circle, so there is some position for
which~$|s_{n+1}-s_n|=h|f_{n+\frac12}|$, so that~$s_{n+1}$
satisfies Equation~\cref{eq:i}.  The only error introduced by discretisation
is that~$c_k$ rotates with an angular frequency less
than~$\omega_k$, bounded by~$\pi/h$ as~$\omega_k$ becomes
large.

In practice, the iterations of~Equation~\cref{eq:vi} must be used
instead of~Equation~\cref{eq:i}, and this is less stable.  For
Hamiltonian dynamics with the form of Equation~\cref{eq:iii}, the
iteration rule becomes
\begin{equation}
s_{n+1}^{(j+1)}=s_n+hf_{n+\frac12}^{(j)}\qquad f_{n+\frac12}^{(j+1)}
	=-{i\over2}Hs_n-{i\over2}Hs_{n+1}^{(j)}
\end{equation}
or
\begin{equation}
s_{n+1}^{(j+1)}=\left(1-{ih\over2}H\right)s_n-{ih\over2}Hs_{n+1}^{(j)}
\end{equation}
with solution
\begin{equation}\label{eq:viii}
s_{n+1}^{(j)}=U^{(j)}s_n
	=\left(1+2\sum_{i=1}^j\left({-ihH\over 2}\right)^j\right)s_n.
\end{equation}
The time step matrix~$U^{(j)}$ for~$j$ iterations has the the same
eigenvectors~$v_k$ as the hamiltonian matrix~$H$.  Each of its
eigenvalues~$u_k^{(j)}$ is the sum of a geometric series
in~$-ih\omega_k/2$, which agrees to second order in~$h\omega_k$ with
the eigenvalue~$e^{-ih\omega_k}$ of the exact time step
operator~$U=e^{-ihH}$.  The relation between~$h\omega_k$ and~$u_k^{(j)}$ is
plotted in \cref{fig:ii}, for the first four iterations.

\begin{figure}[htbp]
  \centering
  \includegraphics[width=0.7\textwidth]{wev.eps}
  \caption{Relation between eigenvectors~$\omega_k$ of~$H$ and~$u_k^{(j)}$ of~$U^{(j)}$.  The lightest line is for~$j=1$, and the heavier ones are for~$j=2$, 3, 4.  The exact time evolution operator~$U$ has unimodular eigenvalues.}
  \label{fig:ii}
\end{figure}

The geometric series for~$U^{(j)}$ will converge if and only if the
angular frequencies~$\omega_k$ are bounded by the reciprocal time
step~$\frac2h$, a kind of Nyquist condition.  If the spectral
hamiltonian matrix has an eigenvalue~$\omega_k\gg \frac2h$, then~$U^{(j)}$
will enlarge the component of~$v_k$ in~$s_n$ by a factor close
to~$2(h\omega_k/2)^j$ over the time step~$h$.  This component is
thus a parasitic solution growing at the
rate~$\frac{\ln2}h+\frac{j}{h}\ln\left(\frac{h\omega_k}2\right)$.

The semi-implicit method is rarely used to solve the linear
Equation~\cref{eq:iii}, and more often used to solve a nonlinear
Gross-Pitaevskii~equation of the form
\begin{equation}
	{\partial\psi\over\partial t}=-i(\nabla^2+V+g|\psi|^2)\psi.
\end{equation}
However, following the usual approach to the stability of integation formulae \cite{1996-Trefethen-finite}, the mapping from~$\psi$ to $\partial\psi\over\partial t$ can be linearised at each time step, as
\begin{equation}
	\left.{\partial\psi\over\partial t}\right|_{\psi=\psi(t)+\delta\psi}
		\approx -i(\nabla^2+V+g|\psi(t)|^2)\psi(t) - i\hat H'(\psi(t))\delta\psi,
\end{equation}
where~$\hat H'(\psi(t))$ is an effective hamiltonian operator for
the field~$\delta\psi$.  This linearisation has been studied
extensively, and the dynamics generated by the hamiltonian~$\hat
H'$ are Bogoliubov sound waves \cite{aop-70-67}.  When this equation
is discretised with the semi-implicit formula, the stability condition
is that the Bogoliubov sound wave modes on a spatial grid of
spacing~$l$ have frequencies bounded by~$2/h$.  For a sufficiently
fine grid, the high-frequency Bogoliubov modes have the same
frequencies as the eigenvectors of~$H$ \cite{2002-Pethick-Bose}.
Most simulations use grids fine enough for this to be the case, and
the nonlinear term will not affect their stability.

\section{Numerical experiments}

As a test of this analysis, a quantum Kerr oscillator was simulated
using the semi-implicit formula.  This system is an analog of the
fields that are usually simulated using the formula, where all the
particles are confined to the same orbital.  The states with definite
numbers of particles are an orthonormal basis for the quantum Hilbert
space, so the coefficient vector~$s$ has indices~$n=0$, 1, \dots,
$N$, where the possible number of particles is limited to~$N$.  In
the example run here, $N=25$.  The hamiltonian matrix is diagonal,
with
\begin{equation}
	H_{nn} = \omega_n = n(n-9)/2.
\end{equation}
The eigenvalues range from~$\omega_4=\omega_5=-10$,
through~$\omega_1=\omega_9=0$, up to~$\omega_{25}=200$.

The oscillator was initialised in a quantum optical coherent state
\cite{prx-131-2766} with a Poisson distribution of particle numbers,
\begin{equation}\label{eq:vii}
	s_n = e^{-|\alpha|^2/2}\frac{\alpha^n}{\sqrt{n!}}.
\end{equation}
The experiment was run with the coherent amplitude~$\alpha=2$.  This
provides some small initial coefficients with large eigenfrequencies,
which demonstrate the growth of the parasitic solutions.

\begin{figure}[htbp]
  \centering
  \label{fig:a}\includegraphics[width=\textwidth]{pev.eps}
  \caption{Growth of parasitic eigenvectors.  Dotted line shows the stability limit $\omega=2/h$, dashed line shows the predicted growth rate~$\frac{i}{h}\ln(\frac{h\omega}2)$.}
  \label{fig:i}
\end{figure}

The results of solving this Hamiltonian with the semi-implicit
method, with 4~iterations and time step~$h=0.03$, are shown in
\cref{fig:a}.  The hamiltonian matrix is diagonal, so the norms~$|s_n|$
are constant in the exact solution.  The right hand side of
Equation~\cref{eq:vii} is shown as the solid lines at~$t=0$ and~$t=0.09$,
and the values of~$|s_n|$ computed by the semi-implicit formula are
shown as the rulings between those lines.  The size~$|s_n|$ is
plotted for all four iterations of the formula within each timestep.
\Cref{eq:viii} shows that, for~$\omega_n\gg 2/h$, the growth of~$s_n$
from step to step is still geometric, resulting in straight lines on the logarithmic
scale of the graph.

Parasitic growth can be
seen at the left hand side, where~$\omega_n$ is large.  The limit
of stability is~$\omega_n=2/h\approx 67$, and the dotted line
indicates~$n=17$, with~$\omega_{17}=68$.  The dashed line shows the result of
growing~$|s_n|$ at the predicted rate~$\frac4h\ln(\frac{h\omega}2)$.
The undulations in the lines~$|s_n(t)|$ correspond to the growth
rates given by the different lines in \cref{fig:ii}, each of which applies
for the initial fraction of each step up to the corresponding iteration.

\Cref{fig:a} shows how abruptly the semi-implicit formula becomes
unstable as the frequency~$\omega_k$ increases through the limit
of stability.  While this might be unfortunate from a practical
point of view, it makes the formula useful for evaluating new ways
to discretise the hamiltonian.  The largest~$\omega_k$ can be
estimated by integrating the dynamics over a few time steps, and
adjusting the time step to find the limit of stability.  Quantum
states inspire some very clever methods of discretisation
\cite{jcp-64-63,jcp-119-1289}, because the state of a quantum system
with many degrees of freedom lies in a vector space of just as many
dimensions, a space where any grid will have impossibly many points.
It is not always straightforward to estimate the eigenvalues of the
discrete hamiltonians that result from these methods.  In particular,
the discrete hamiltonian can be a non-normal operator, in which
case quite delicate arguments about pseudospectra would be required
to rigorously determine the stability of the discretised dynamics.
Running those dynamics, and looking out for parasitic solutions,
is a fast and simple alternative.

\section{Conclusion}

The semi-implicit formula discussed here is only one of the formulae
that can be used to integrate ordinary differential equations, and
that could be applied to solve Equation~\cref{eq:iii} numerically.  However,
most of these formulae suffer from similar problems, and become
unstable when the eigenvalues of the hamiltonian matrix are large
compared to the time step.  The phenomenon of stiffness in discretised
partial differential equations is well known \cite{something}.

There are special formulae to integrate stiff differential equations,
which remain stable when the time steps are long compared to the
eigenvalues of~$H$.  However, these remain stable by ignoring any
physics that is happening on the fast timescales corresponding to
the large eigenvalues.  If the goal is to simulate physics at a
given frequency, then it seems reasonable that the simulation will
require a time step around the corresponding Nyquist rate.

On the other hand, stiff integration formulae would be appropriate
where the dynamics being simulated is relatively slow, but the
representation of the quantum state inadvertently contains some
components with large energies whose phases oscillate rapidly.  This
was the case in the example shown above, where the coefficient
vector included coefficients with large~$n$, for which~$s_n$ was
too small to matter.  However, $\omega_n$ was large enough that the
formula caused these minute components to grow and dominate the
actual solution.  This could be addressed by using an integration
formula that was adapted to stiff differential equations.  But it
would be better to change the representation to exclude the high
energy components, or to modify the simulated dynamics so that these
components oscillate with reasonable frequencies.

The example above can be modified very simply, by reducing~$N$ to
exclude the high frequencies.  An expansion with coefficients~$s_0$
to~$s_{17}$ would be stable with the time step used above, and the
coefficients~$s_n$ with $n>17$ are small.  However, this is a
contrived example.  Since the energy eigenstates are known, their
amplitudes can be calculated directly as complex
exponentials~$e^{i\omega_nt}$, and there is no need to use any
integration formula, stable or otherwise.

When the equation being solved is simply Schr\"odinger's equation
with repulsion, there are ways to modify the semi-implicit method
and avoid the large energies.  The most commonly used is an interaction
picture method \cite{2000-CaradocDavies-vortex}.  This is a split
step method, where a Fourier transform~$\psi(t,k)$ is taken at each
time step, and the kinetic energy term proportional to~$-i\nabla^2\psi$
is evaluated exactly as~$\psi(t+h,k)=e^{ik^2h}\psi(t,k)$.  This
ensures that~$\|\psi(t+h)\|=\|\psi(t)\|$, so that parasitic modes
can not grow only because of their high kinetic energy.

Interaction picture methods are very effective when they work.
However, the nonlinear terms can also become large, and this occurs
quite often in practice.  For example, it is often desirable to
simulate a cold gas in a harmonic trap, with~$V(x)\propto x^2$.  In
this case, the energy of the gas oscillates between potential and
kinetic, and the potential energy of a particle far from the origin
can be as great as the kinetic energy of a particle with large
momentum.  This causes the equation to remain just as stiff even
when the kinetic energy term is solved exactly.

In a real simulation, the energy eigenstates are not known beforehand,
and the quantum state is represented some other way.  This could
be an expansion over a set of states that are not energy eigenstates,
a variational ansatz, or a similar method.  Some of these representations
will allow states with spuriously high energies to be represented,
so that the differential equations satisfied by the representation
are unnecessarily stiff.  In this case, care must be taken to avoid
inadvertently representing high energy states, or to modify the
Hamiltonian dynamics to prevent them from causing stiffness.  Details
for one such representation, using Gaussian wave packets, will be
given in a future paper.

\section*{Acknowledgments}
I would like to acknowledge the support of Swinburne University of Technology.  Thanks to Run Yan Teh for useful suggestions.

\bibliographystyle{siamplain}
\bibliography{references,goats}
\end{document}
