% SIAM Article Template
\documentclass[review]{siamart1116}

% Packages and macros go here
\usepackage{lipsum}
\usepackage{amsfonts}
\usepackage{graphicx}
\usepackage{epstopdf}
\usepackage{algorithmic}
\ifpdf
  \DeclareGraphicsExtensions{.eps,.pdf,.png,.jpg}
\else
  \DeclareGraphicsExtensions{.eps}
\fi

%strongly recommended
\numberwithin{theorem}{section}

% Declare title and authors, without \thanks
\newcommand{\TheTitle}{Stability of the iterative semi-implicit formula for quantum dynamics} 
\newcommand{\TheAuthors}{Rodney E. S. Polkinghorne}

% Sets running headers as well as PDF title and authors
\headers{\TheTitle}{\TheAuthors}

% Title. If the supplement option is on, then "Supplementary Material"
% is automatically inserted before the title.
\title{{\TheTitle}\thanks{Submitted to the editors DATE.
\funding{This work was funded by the Fog Research Institute under contract no.~FRI-454.}}}

% Authors: full names plus addresses.
\author{
  Rodney E. S. Polkinghorne\thanks{Swinburne University of Technology
    (\email{rpolkinghorne@swin.edu.au}, \url{http://www.imag.com/\string~ddoe/}).}
}

\usepackage{amsopn}
\DeclareMathOperator{\diag}{diag}


%%% Local Variables: 
%%% mode:latex
%%% TeX-master: "ex_article"
%%% End: 

% Optional PDF information
\ifpdf
\hypersetup{
  pdftitle={\TheTitle},
  pdfauthor={\TheAuthors}
}
\fi

% The next statement enables references to information in the
% supplement. See the xr-hyperref package for details.

\externaldocument{ex_supplement}

% FundRef data to be entered by SIAM
%<funding-group>
%<award-group>
%<funding-source>
%<named-content content-type="funder-name"> 
%</named-content> 
%<named-content content-type="funder-identifier"> 
%</named-content>
%</funding-source>
%<award-id> </award-id>
%</award-group>
%</funding-group>

\begin{document}

\maketitle

% REQUIRED
\begin{abstract}
  This is an example SIAM \LaTeX\ article. This can be used as a
  template for new articles.  Abstracts must be able to stand alone
  and so cannot contain citations to the paper's references,
  equations, etc.  An abstract must consist of a single paragraph and
  be concise. Because of online formatting, abstracts must appear as
  plain as possible. Any equations should be inline.
\end{abstract}

% REQUIRED
\begin{keywords}
  example, \LaTeX
\end{keywords}

% REQUIRED
\begin{AMS}
  68Q25, 68R10, 68U05
\end{AMS}

\section{Introduction}

The semi-implicit formula is a rule for integrating systems of
ordinary differential equations \cite{jkp-132-312,jkp-93-144}.  It
is often used as a method of lines, to solve stochastic partial
differential equations that occur when phase space Monte-Carlo
methods are used in quantum optics and in atom optics \cite{pra-58-4824}.
This is done as a method of lines, in which the spatial axes are
discretised, often by spectral derivatives, to give a system of
ordinary differential equations \cite{2000-Trefethen-Spectral}.
The semi-implicit formula is then used to integrate these.

{\bf Relate Hilbert space expansions to methods of lines.}

For a system of equations~${dx\over dt}=f(x,t)$, with time step~$h$,
the formula is
\begin{equation}\label{eq:i}
    x_{n+1}=x_n+hf_{n+\frac12},
\end{equation}
where
\begin{equation}\label{eq:ii}
    f_{n+\frac12}=f\left({x_n+x_{n+1}\over 2}\right).
\end{equation}

The formula has the theoretical advantage that it conserves Hilbert space norm when it is applied to a quantum equation of motion of the form
\begin{equation}\label{eq:iii}
	{dx\over dt}=-iHx,
\end{equation}
where $H$ is a hermitian operator.  However, it is mostly used because it has been found to work well in practice.

It is not obvious how to apply this formula, because of the need
to evaluate~$x_{n+1}$ and~$f_{n+\frac12}$ self-consistently.  The
usual way of doing this is a very simple iterative one.  At first,
$x_{n+1}$ is set to~$x_n$, and~$f_{n+\frac12}$ to~$f'(x_n)$.  Then
Equations~\cref{eq:i,eq:ii} are evaluated in turn, and the values of~$x_n$
and~$f_{n+\frac12}$ updated.  This ought to continue until they
converge, but in practice this is continued for a fixed number of
iterations.  Formally, this process gives a sequence of~$x_{n+1}^j$
and~$f_{n+\frac12}^j$, with
\begin{subequations}
\begin{align}
	\label{eq:iv} x_{n+1}^0&=x_n\qquad f_{n+\frac12}^0=f(x_n)\\
	\label{eq:v} x_{n+1}^{j+1}&=x_n+hf_{n+\frac12}^j\qquad f_{n+\frac12}^{j+1}=f\left({x_n+x_{n+1}^j\over 2}\right),\qquad j\ge0.
\end{align}
\end{subequations}
Any fixed point of this iteration is a self-consistent solution to
Equations~\cref{eq:i,eq:ii}, but the question remains of when the
iteration converges.

It is a generally known problem that the time steps in formulae for
partial differential equations need to be reduced in proportion to
the spatial grid steps, in order for the formulae to remain stable.
To date, no one has made a specific analysis of the iterative
semi-implicit formula, so the stable time step has had to be
determined by trial and error.  This can be a time consuming process
for fields that are being simulated on large grids.

This paper analyses the semi-implicit formula, in the case that it
is used to integrate linear Hamiltonian dynamics as in
Equation~\cref{eq:iii}.  The main result is that the stability
depends on a Nyquist-like condition, and the iterations converge
if and only if the largest eigenvalue of~$H$ is no more than~$2/h$.

\section{Stability analysis}

Suppose that the equation being integrated has the form of
Equation~\cref{eq:iii}, where~$x$ is a vector of coefficients
obtained by expanding a ket~$|\psi\rangle$ over an orthonormal
basis, and~$f$ is the linear operator~$-iH$, where~$H$ is the matrix
for the Hamiltonian operator in this basis.  The
iteration rules of Equation~\cref{eq:v} become
$$x_{n+1}^{j+1}=x_n+hf_{n+\frac12}^j\qquad f_{n+\frac12}^{j+1}
	=-{i\over2}Hx_n-{i\over2}Hx_{n+1}^j$$
or
$$x_{n+1}^{j+1}=\left(1-{ih\over2}H\right)x_n-{ih\over2}Hx_{n+1}^j$$
with solution
$$x_{n+1}^j=U_jx_n=\left(1-ihH-\frac{(hH)^2}2
	+{i(hH)^3\over 4}+\cdots+2\left({-ihH\over 2}\right)^j\right)x_n.$$
The time step matrix~$U_j$ has the the same eigenvectors~$\omega_m$ as the
hamiltonian matrix~$H$.  The eigenvalues of~$U_j$ are the sums of a geometric series in~$\omega_m$, with the relation between~$\omega_m$ and the sum~$\omega_m^j$ plotted in \cref{fig:ii}.

\begin{figure}[htbp]
  \centering
  \includegraphics[width=\textwidth]{wev.eps}
  \caption{Relation between eigenvectors~$\omega_m$ of~$H$ and~$\omega_m^j$ of~$U_j$.  The lightest line is for~$j=1$, and they get heavier for~$j=2$, 3, 4.  The exact time evolution operator~$U$ is unitary, with unimodular eigenvalues.}
  \label{fig:ii}
\end{figure}

The sequence of~$U_j$ is a power series, which will converge if and only if the angular frequencies~$\omega_m$ are bounded by~$2/h$, a kind of Nyquist
condition.  If the discretised Hamiltonian has an eigenstate~$x_\omega$
with a frequency~$\omega\gg 2/h$, then~$U$ will magnify the component
of~$x_\omega$ in~$x$ by a factor close to~$(h\omega/2)^i$, and this
component is a parasitic solution growing steadily at the
rate~$\frac{i}{h}\ln(\frac{h\omega}2)$.

The semi-implicit method is rarely used to solve the linear Equation~\cref{eq:iii}, and more often used to solve a nonlinear Gross-Pitaevskii~equation of the form
\begin{equation}
	{\partial\psi\over\partial t}=-i(\hat H+g|\psi|^2)\psi.
\end{equation}
However, following the usual approach to the stability of integation formulae \cite{something}, the mapping
$\psi\to -i(\hat H+g|\psi|^2)\psi$ can be linearised at each time step, in the form
\begin{equation}
	\psi(t)+\delta\psi\to -i(\hat H+g|\psi(t)|^2)\psi(t) -i\hat H'\delta\psi.
\end{equation}
This linearisation has been extensively studied, and the operator~$\hat
H'$ is the hamiltonian for Bogoliubov sound waves \cite{something}.
The stability conditon for the Gross-Pitaevskii~equation is that
the highest frequency of the Bogoliubov sound waves in the discretised
Hamiltonian is less than~$2/h$.  In the limit of fine grids, the
high-frequency Bogoliubov sound waves have the same frequencies as
the eigenstates of~$\hat H$.  Most simulations use grids fine enough
for this to be the case, and the nonlinear term will not affect the
stability of such simulations.

\section{Numerical experiments}

\begin{figure}[htbp]
  \centering
  \label{fig:a}\includegraphics[width=\textwidth]{pev.eps}
  \caption{Growth of parasitic eigenvectors.  Dotted line shows the stability limit $\omega=2/h$, dashed line shows the predicted growth rate~$\frac{i}{h}\ln(\frac{h\omega}2)$.}
  \label{fig:i}
\end{figure}

\section{Conclusion}

The semi-implicit formula considered here is a special case, but most ODE formulae will become unstable when frequencies get large compared to time steps.

Stiff formulae stay stable, but reject the physics that is happening on fast timescales.  If we want to simulate physics at frequency~$\omega$, it seems reasonable to require time steps~$?$.

The problem comes when the interesting physics occurs at slow timescales, but the state representation is inadvertently capable of representing high energies that evolve on fast timescales.

In the case here, where we expanded over energy eigenstates, you can simply truncate.  This is contrived: if you know the energy eigenstates, you already know the answer, and you don't need to simulate anything.

In general, you will use a variational ansatz, or some other tricky representation.  In this case, some care must be taken to avoid inadvertently representing high energy states.  Details to come in future papers.

\section*{Acknowledgments}
We would like to acknowledge the assistance of volunteers in putting
together this example manuscript and supplement.

\bibliographystyle{siamplain}
\bibliography{references}
\end{document}
