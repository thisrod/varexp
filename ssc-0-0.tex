% SIAM Article Template
\documentclass[review]{siamart1116}

% Packages and macros go here
\usepackage{lipsum}
\usepackage{amsfonts}
\usepackage{graphicx}
\usepackage{epstopdf}
\usepackage{algorithmic}
\ifpdf
  \DeclareGraphicsExtensions{.eps,.pdf,.png,.jpg}
\else
  \DeclareGraphicsExtensions{.eps}
\fi

%strongly recommended
\numberwithin{theorem}{section}

% Declare title and authors, without \thanks
\newcommand{\TheTitle}{Stability of the semi-implicit formula for quantum dynamics} 
\newcommand{\TheAuthors}{Rodney E. S. Polkinghorne}

% Sets running headers as well as PDF title and authors
\headers{\TheTitle}{\TheAuthors}

% Title. If the supplement option is on, then "Supplementary Material"
% is automatically inserted before the title.
\title{{\TheTitle}\thanks{Submitted to the editors DATE.
\funding{This work was funded by the Fog Research Institute under contract no.~FRI-454.}}}

% Authors: full names plus addresses.
\author{
  Rodney E. S. Polkinghorne\thanks{Swinburne University of Technology
    (\email{rpolkinghorne@swin.edu.au}, \url{http://www.imag.com/\string~ddoe/}).}
}

\usepackage{amsopn}
\DeclareMathOperator{\diag}{diag}


%%% Local Variables: 
%%% mode:latex
%%% TeX-master: "ex_article"
%%% End: 

% Optional PDF information
\ifpdf
\hypersetup{
  pdftitle={\TheTitle},
  pdfauthor={\TheAuthors}
}
\fi

% The next statement enables references to information in the
% supplement. See the xr-hyperref package for details.

\externaldocument{ex_supplement}

% FundRef data to be entered by SIAM
%<funding-group>
%<award-group>
%<funding-source>
%<named-content content-type="funder-name"> 
%</named-content> 
%<named-content content-type="funder-identifier"> 
%</named-content>
%</funding-source>
%<award-id> </award-id>
%</award-group>
%</funding-group>

\begin{document}

\maketitle

% REQUIRED
\begin{abstract}
  This is an example SIAM \LaTeX\ article. This can be used as a
  template for new articles.  Abstracts must be able to stand alone
  and so cannot contain citations to the paper's references,
  equations, etc.  An abstract must consist of a single paragraph and
  be concise. Because of online formatting, abstracts must appear as
  plain as possible. Any equations should be inline.
\end{abstract}

% REQUIRED
\begin{keywords}
  example, \LaTeX
\end{keywords}

% REQUIRED
\begin{AMS}
  68Q25, 68R10, 68U05
\end{AMS}

\section{Introduction}

The semi-implicit method \cite{jkp-132-313,jkp-93-144} is often used to solve stochastic partial differential equations that arise when Monte-Carlo methods are used in quantum optics and in atom optics.  This is done as a method of lines, in which the spatial axes are discretised, often by spectral derivatives, to give a system of ordinary differential equations.  The semi-implicit formula is then used to integrate these.  For a system of equations~${dx\over dt}=f(x,t)$, with time step~$h$, the formula is
\begin{equation}\label{eq:i}
    x_{n+1}=x_n+hf_{n+\frac12},
\end{equation}
where
\begin{equation}\label{eq:ii}
    f_{n+\frac12}=f\left({x_n+x_{n+1}\over 2}\right).
\end{equation}

The formula has the theoretical advantage that it conserves Hilbert space norm when it is applied to a quantum equation of motion of the form
\begin{equation}\label{eq:iii}
	{dx\over dt}=-iHx,
\end{equation}
where $H$ is a hermitian operator.  However, it is mostly used because it has been found to work well in practice.

It is not obvious how to apply this formula, because of the need
to evaluate~$x_{n+1}$ and~$f_{n+\frac12}$ self-consistently.  The
usual way of doing this is a very simple iterative one.  At first,
$x_{n+1}$ is set to~$x_n$, and~$f_{n+\frac12}$ to~$f'(x_n)$.  Then
Equations~\cref{eq:i,eq:ii} are evaluated in turn, and the values of~$x_n$
and~$f_{n+\frac12}$ updated.  This ought to continue until they
converge, but in practice this is continued for a fixed number of
iterations.  Formally, this process gives a sequence of~$x_{n+1}^i$
and~$f_{n+\frac12}^i$, with
\begin{subequations}
\begin{align}
	\label{eq:iv} x_{n+1}^0&=x_n\qquad f_{n+\frac12}^0=f(x_n)\\
	\label{eq:v} x_{n+1}^{i+1}&=x_n+hf_{n+\frac12}^i\qquad f_{n+\frac12}^{i+1}=f\left({x_n+x_{n+1}^i\over 2}\right).
\end{align}
\end{subequations}
Any fixed point of this iteration is a self-consistent solution to
Equations~\cref{eq:i,eq:ii}, but the question remains of when the
iteration converges.

It is a generally known problem that the time steps in formulae for
partial differential equations need to be reduced in proportion to
the spatial grid steps, in order for the formulae to remain stable.
To date, no one has made a specific analysis of the iterative
semi-implicit formula, so the stable time step has had to be
determined by trial and error.  This can be a time consuming process
for fields that are being simulated on large grids.

This paper analyses the semi-implicit formula, in the case that it
is used to integrate linear Hamiltonian dynamics as in
Equation~\cref{eq:iii}.  The main result is that the stability
depends on a Nyquist-like condition, and the iterations converge
if and only if the largest eigenvalue of~$H$ is no more than~$2/h$.

\section{Stability analysis}

Suppose that the equation being integrated has the form of
Equation~\cref{eq:iii}, where~$x$ is a vector of coefficients
obtained by expanding a ket~$|\psi\rangle$ over an orthonormal
basis, and~$f$ is the linear operator~$-iH$, where~$H$ is the matrix
for the Hamiltonian operator in this basis.  The
iteration rules of Equation~\cref{eq:v} becomes
$$x_{n+1}^{i+1}=x_n+hf_{n+\frac12}^i\qquad f_{n+\frac12}^{i+1}
	=-{i\over2}Hx_n-{i\over2}Hx_{n+1}^i$$
or
$$x_{n+1}^{i+1}=(1-{ih\over2}H)x_n-{ih\over2}Hx_{n+1}^i$$
with solution
$$x_{n+1}^i=Ux_n=\left(1-ihH-\frac{(hH)^2}2
	+{i(hH)^3\over 4}+\cdots+2\left({-ihH\over 2}\right)^i\right)x_n.$$
The time step matrix~$U$ has the the same eigenvectors as the
hamiltonian matrix~$H$.

The iteration will converge if and only if the angular frequencies
that are eigenvalues of~$H$ are bounded by~$2/h$, a kind of Nyquist
condition.  If the discretised Hamiltonian has an eigenstate~$x_\omega$
with a frequency~$\omega\gg 2/h$, then~$U$ will magnify the component
of~$x_\omega$ in~$x$ by a factor close to~$(h\omega/2)^i$, and this
component is a parasitic solution growing steadily at the
rate~$(h\omega/2)^i/h$.

The semi-implicit method is rarely used to solve the linear Equation~\cref{eq:iii}, and more often used to solve a nonlinear Gross-Pitaevskii~equation of the form
\begin{equation}
	{\partial\psi\over\partial t}=-i(\hat H+g|\psi|^2)\psi.
\end{equation}
However, following the usual approach to the stability of integation formulae \cite{something}, the mapping
$\psi\to -i(\hat H+g|\psi|^2)\psi$ can be linearised at each time step, in the form
\begin{equation}
	\psi(t)+\delta\psi\to -i(\hat H+g|\psi(t)|^2)\psi(t) -i\hat H'\delta\psi.
\end{equation}
This linearisation has been extensively studied, and the operator~$\hat
H'$ is the hamiltonian for Bogoliubov sound waves \cite{something}.
The stability conditon for the Gross-Pitaevskii~equation is that
the highest frequency of the Bogoliubov sound waves in the discretised
Hamiltonian is less than~$2/h$.  In the limit of fine grids, the
high-frequency Bogoliubov sound waves have the same frequencies as
the eigenstates of~$\hat H$.  Most simulations use grids fine enough
for this to be the case, and the nonlinear term will not affect the
stability of such simulations.




\section{Main results}
\label{sec:main}

We interleave text filler with some example theorems and theorem-like
items.

\lipsum[4]

Here we state our main result as \cref{thm:bigthm}; the proof is
deferred to \cref{sec:proof}.

\begin{theorem}[$LDL^T$ Factorization \cite{GoVa13}]\label{thm:bigthm}
  If $A \in \mathbb{R}^{n \times n}$ is symmetric and the principal
  submatrix $A(1:k,1:k)$ is nonsingular for $k=1:n-1$, then there
  exists a unit lower triangular matrix $L$ and a diagonal matrix
  \begin{displaymath}
    D = \diag(d_1,\dots,d_n)
  \end{displaymath}
  such that $A=LDL^T$. The factorization is unique.
\end{theorem}

\lipsum[6]

\begin{theorem}[Mean Value Theorem]\label{thm:mvt}
  Suppose $f$ is a function that is continuous on the closed interval
  $[a,b]$.  and differentiable on the open interval $(a,b)$.
  Then there exists a number $c$ such that $a < c < b$ and
  \begin{displaymath}
    f'(c) = \frac{f(b)-f(a)}{b-a}.
  \end{displaymath}
  In other words,
  \begin{displaymath}
    f(b)-f(a) = f'(c)(b-a).
  \end{displaymath}
\end{theorem}

Observe that \cref{thm:bigthm,thm:mvt,cor:a} correctly mix references
to multiple labels.

\begin{corollary}\label{cor:a}
  Let $f(x)$ be continuous and differentiable everywhere. If $f(x)$
  has at least two roots, then $f'(x)$ must have at least one root.
\end{corollary}
\begin{proof}
  Let $a$ and $b$ be two distinct roots of $f$.
  By \cref{thm:mvt}, there exists a number $c$ such that
  \begin{displaymath}
    f'(c) = \frac{f(b)-f(a)}{b-a} = \frac{0-0}{b-a} = 0.
  \end{displaymath}
\end{proof}

Note that it may require two \LaTeX\ compilations for the proof marks
to show.

Display matrices can be rendered using environments from \texttt{amsmath}:
\begin{equation}\label{eq:matrices}
S=\begin{bmatrix}1&0\\0&0\end{bmatrix}
\quad\text{and}\quad
C=\begin{pmatrix}1&1&0\\1&1&0\\0&0&0\end{pmatrix}.
\end{equation}
Equation \cref{eq:matrices} shows some example matrices.

We calculate the Fr\'{e}chet derivative of $F$ as follows:
\begin{subequations}
\begin{align}
  F'(U,V)(H,K) 
  &= \langle R(U,V),H\Sigma V^{T} + U\Sigma K^{T} -
  P(H\Sigma V^{T} + U\Sigma K^{T})\rangle \label{eq:aa} \\
  &= \langle R(U,V),H\Sigma V^{T} + U\Sigma K^{T}\rangle 
  \nonumber \\
  &= \langle R(U,V)V\Sigma^{T},H\rangle + 
  \langle \Sigma^{T}U^{T}R(U,V),K^{T}\rangle. \label{eq:bb}
\end{align}
\end{subequations}
\Cref{eq:aa} is the first line, and \cref{eq:bb} is the last line.

\section{Algorithm}
\label{sec:alg}

\lipsum[40]

Our analysis leads to the algorithm in \cref{alg:buildtree}.

\begin{algorithm}
\caption{Build tree}
\label{alg:buildtree}
\begin{algorithmic}
\STATE{Define $P:=T:=\{ \{1\},\ldots,\{d\}$\}}
\WHILE{$\#P > 1$}
\STATE{Choose $C^\prime\in\mathcal{C}_p(P)$ with $C^\prime := \operatorname{argmin}_{C\in\mathcal{C}_p(P)} \varrho(C)$}
\STATE{Find an optimal partition tree $T_{C^\prime}$ }
\STATE{Update $P := (P{\setminus} C^\prime) \cup \{ \bigcup_{t\in C^\prime} t \}$}
\STATE{Update $T := T \cup \{ \bigcup_{t\in\tau} t : \tau\in T_{C^\prime}{\setminus} \mathcal{L}(T_{C^\prime})\}$}
\ENDWHILE
\RETURN $T$
\end{algorithmic}
\end{algorithm}

\lipsum[41]

\section{Experimental results}
\label{sec:experiments}

\lipsum[50]

\Cref{fig:testfig} shows some example results. Additional results are
available in the supplement in \cref{tab:foo}.

\begin{figure}[htbp]
  \centering
  \label{fig:a}\includegraphics{lexample_fig1}
  \caption{Example figure using external image files.}
  \label{fig:testfig}
\end{figure}

\lipsum[51]

\section{Discussion of \texorpdfstring{{\boldmath$Z=X \cup Y$}}{Z = X union Y}}

\lipsum[76]

\section{Conclusions}
\label{sec:conclusions}

Some conclusions here.


\appendix
\section{An example appendix} 
\lipsum[71]

\section*{Acknowledgments}
We would like to acknowledge the assistance of volunteers in putting
together this example manuscript and supplement.

\bibliographystyle{siamplain}
\bibliography{references}
\end{document}
