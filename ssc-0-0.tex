% SIAM Article Template
\documentclass[review]{siamart1116}

% Packages and macros go here
\usepackage{lipsum}
\usepackage{amsfonts}
\usepackage{graphicx}
\usepackage{epstopdf}
\usepackage{algorithmic}
\ifpdf
  \DeclareGraphicsExtensions{.eps,.pdf,.png,.jpg}
\else
  \DeclareGraphicsExtensions{.eps}
\fi

%strongly recommended
\numberwithin{theorem}{section}

% Declare title and authors, without \thanks
\newcommand{\TheTitle}{Stability of the iterative semi-implicit formula for quantum dynamics} 
\newcommand{\TheAuthors}{Rodney E. S. Polkinghorne}

% Sets running headers as well as PDF title and authors
\headers{\TheTitle}{\TheAuthors}

% Title. If the supplement option is on, then "Supplementary Material"
% is automatically inserted before the title.
\title{{\TheTitle}\thanks{Submitted to the editors DATE.
\funding{This work was funded by the Fog Research Institute under contract no.~FRI-454.}}}

% Authors: full names plus addresses.
\author{
  Rodney E. S. Polkinghorne\thanks{Swinburne University of Technology
    (\email{rpolkinghorne@swin.edu.au}, \url{http://www.imag.com/\string~ddoe/}).}
}

\usepackage{amsopn}
\DeclareMathOperator{\sinc}{sinc}


%%% Local Variables: 
%%% mode:latex
%%% TeX-master: "ex_article"
%%% End: 

% Optional PDF information
\ifpdf
\hypersetup{
  pdftitle={\TheTitle},
  pdfauthor={\TheAuthors}
}
\fi

% The next statement enables references to information in the
% supplement. See the xr-hyperref package for details.

\externaldocument{ex_supplement}

% FundRef data to be entered by SIAM
%<funding-group>
%<award-group>
%<funding-source>
%<named-content content-type="funder-name"> 
%</named-content> 
%<named-content content-type="funder-identifier"> 
%</named-content>
%</funding-source>
%<award-id> </award-id>
%</award-group>
%</funding-group>

\begin{document}

\maketitle

% REQUIRED
\begin{abstract}
  This is an example SIAM \LaTeX\ article. This can be used as a
  template for new articles.  Abstracts must be able to stand alone
  and so cannot contain citations to the paper's references,
  equations, etc.  An abstract must consist of a single paragraph and
  be concise. Because of online formatting, abstracts must appear as
  plain as possible. Any equations should be inline.
\end{abstract}

% REQUIRED
\begin{keywords}
  example, \LaTeX
\end{keywords}

% REQUIRED
\begin{AMS}
  68Q25, 68R10, 68U05
\end{AMS}

\section{Introduction}

% formula equations
The semi-implicit formula is a rule for integrating systems of
ordinary differential equations \cite{jkp-132-312,jkp-93-144}.  It
is often used as a method of lines, to solve stochastic partial
differential equations that occur when phase space Monte-Carlo
methods are used in quantum optics and in atom optics \cite{pra-58-4824}.
These equations are related to Schr\"odinger's equation,
\begin{equation}\label{eq:vi}
    \frac{\partial\psi}{\partial t} = -i(-\nabla^2 + V)\psi,
\end{equation}
with additional nonlinear and random terms.  The spatial axes are
discretised, often using spectral derivative operators, to give a
system of ordinary or stochastic differential equations
\cite{2000-Trefethen-Spectral}.  Then these equations are integrated
by the semi-implicit formula.

For a system of equations~${dx\over dt}=f(x,t)$,
the semi-implicit formula with time step~$h$ is
\begin{equation}\label{eq:i}
    x_{n+1}=x_n+hf_{n+\frac12},
\end{equation}
where
\begin{equation}\label{eq:ii}
    f_{n+\frac12}=f\left({\textstyle\frac{x_n+x_{n+1}}{2}}\right).
\end{equation}
When this formula is applied to equations of a form similar to~\cref{eq:vi},
\begin{equation}\label{eq:iii}
	{dx\over dt}=-iHx,
\end{equation}
where~$H$ is a hermitian operator acting on the complex vector~$x$,
there is an advantage in principle because the norm of~$x$ is
conserved in the discretised solution.  {\bf Expand on this.}  However, the formula is
used mainly because it has been found to work well in practice.

% iterations
It is not obvious how to implement this formula, because~$x_{n+1}$
and~$f_{n+\frac12}$ need to be evaluated self-consistently.  The
usual way is a simple iteration.  To start,
$x_{n+1}$~is initialised to~$x_n$, and~$f_{n+\frac12}$ to~$f'(x_n)$.  Then each iteration
evaluates Equations~\cref{eq:i,eq:ii} in turn, and updates the values
of~$x_{n+1}$ and~$f_{n+\frac12}$.  More precisely, the iterations generate a sequence of~$x_{n+1}^j$
and~$f_{n+\frac12}^j$, with
\begin{subequations}
\begin{align}
	\label{eq:iv} x_{n+1}^0&=x_n\qquad f_{n+\frac12}^0=f(x_n)\\
	\label{eq:v} x_{n+1}^{j+1}&=x_n+hf_{n+\frac12}^j
		\qquad f_{n+\frac12}^{j+1}=f\left({\textstyle\frac{x_n+x_{n+1}^j}{2}}\right),\qquad j\ge0.
\end{align}
\end{subequations}
This continues until the sequences converge, which is usually
achieved after a small, fixed number of iterations.  Any fixed point
of the iteration is a self-consistent solution to
Equations~\cref{eq:i,eq:ii}.  The remaining question is under what
conditions the iterations converge.

% steps stability
Generally, when a partial differential equation is discretised by
a given formula, the time step needs to be reduced in proportion
to the grid spacing in order for the discretised solution to remain
stable \cite{something}.  No one has analysed the stability of the
semi-implicit formula with iterations.  The stable time step has
had to be determined by trial and error, a time consuming process
for three-dimensional wave functions that are being simulated on
large grids.

This paper analyses the semi-implicit formula, in the case that it
is used to integrate linear Hamiltonian dynamics as in
Equation~\cref{eq:iii}.  The main result is that the stability
depends on a Nyquist-like condition, and the iterations converge
if and only if the largest eigenvalue of~$H$ is no more than~$2/h$.

\section{Stability analysis}

% basis coefficients
When the semi-implicit rule is applied to \cref{eq:vi}, or a
stochastic or nonlinear variant, the laplacian operator is usually
evaluated by a spectral method.  This is equivalent to expanding
the wave function~$\psi(q)$ over a basis of sinc wave packets
\begin{equation}
	\psi(q) = \sum x_i\phi_i(q),
\end{equation}
{\bf figure out range of $i$}
where
\begin{equation}
	\phi_i(q) = \frac1{\sqrt{l}} \sinc(\textstyle\frac{q}{l} - i).
\end{equation}
This paper uses the convention
\begin{equation}
	\sinc{q} = \frac{\sin(\pi q)}{\pi q},
\end{equation}
under which the functions~$\phi_i$ are an orthonormal basis for the
space of functions limited to bandwidth~$1/2l$.  The spectral method
reduces \cref{eq:vi} to an equation of the form~\cref{eq:iii} for
the coefficient vector~$x$, where
\begin{equation}
	H_{ij} = \int \phi_i(q) (-\nabla^2 + V(q)) \phi_j^\ast(q).
\end{equation}

At this point, the
iteration rules of Equation~\cref{eq:v} become
\begin{equation}
x_{n+1}^{j+1}=x_n+hf_{n+\frac12}^j\qquad f_{n+\frac12}^{j+1}
	=-{i\over2}Hx_n-{i\over2}Hx_{n+1}^j
\end{equation}
or
\begin{equation}
x_{n+1}^{j+1}=\left(1-{ih\over2}H\right)x_n-{ih\over2}Hx_{n+1}^j
\end{equation}
with solution {\bf check factor of 2}
\begin{equation}\label{eq:viii}
x_{n+1}^j=U_jx_n=\left(1-ihH-\frac{(hH)^2}2
	+{i(hH)^3\over 4}+\cdots+2\left({-ihH\over 2}\right)^j\right)x_n.
\end{equation}
The time step matrix~$U_j$ has the the same eigenvectors~$\omega_m$
as the hamiltonian matrix~$H$.  The eigenvalues of~$U_j$ are the
sums of a geometric series in~$\omega_m$, with the relation
between~$\omega_m$ and the sum~$\omega_m^j$ plotted in \cref{fig:ii}.

\begin{figure}[htbp]
  \centering
  \includegraphics[width=\textwidth]{wev.eps}
  \caption{Relation between eigenvectors~$\omega_m$ of~$H$ and~$h\omega_m^j$ of~$U_j$.  The lightest line is for~$j=1$, and they get heavier for~$j=2$, 3, 4.  The exact time evolution operator~$U$ is unitary, with unimodular eigenvalues.}
  \label{fig:ii}
\end{figure}

The sequence of~$U_j$ is a power series, which will converge if and
only if the angular frequencies~$\omega_m$ are bounded by the reciprocal time step~$2/h$, a
kind of Nyquist condition.  If the discretised hamiltonian has an
eigenstate~$x_\omega$ with a frequency~$\omega\gg 2/h$, then~$U$
will magnify the component of~$x_\omega$ in~$x$ by a factor close
to~$2(h\omega/2)^i$, and this component is a parasitic solution
growing steadily at the rate~$\frac{\ln2}h\frac{i}{h}\ln(\frac{h\omega}2)$.

{\bf Rewrite for clarity.}
The semi-implicit method is rarely used to solve the linear
Equation~\cref{eq:iii}, and more often used to solve a nonlinear
Gross-Pitaevskii~equation of the form
\begin{equation}
	{\partial\psi\over\partial t}=-i(\hat H+g|\psi|^2)\psi.
\end{equation}
However, following the usual approach to the stability of integation formulae \cite{something}, the mapping
$\psi\to -i(\hat H+g|\psi|^2)\psi$ can be linearised at each time step, in the form
\begin{equation}
	\psi(t)+\delta\psi\to -i(\hat H+g|\psi(t)|^2)\psi(t) -i\hat H'\delta\psi.
\end{equation}
This linearisation has been extensively studied, and the operator~$\hat
H'$ is the hamiltonian for Bogoliubov sound waves \cite{something}.
The stability conditon for the Gross-Pitaevskii~equation is that
the highest frequency of the Bogoliubov sound waves in the discretised
Hamiltonian is less than~$2/h$.  In the limit of fine grids, the
high-frequency Bogoliubov sound waves have the same frequencies as
the eigenstates of~$\hat H$.  Most simulations use grids fine enough
for this to be the case, and the nonlinear term will not affect the
stability of such simulations.

\section{Numerical experiments}

As a test of this analysis, a quantum Kerr oscillator was simulated
using the semi-implicit method.  This system is a single-orbital
analog of the fields for repulsive particles that are simulated in
quantum optics.  The states with definite numbers of particles are
an orthonormal basis for the quantum Hilbert space, so the coefficient
vector~$x$ has indices~$n=0$, 1, \dots, $N$, where the number of
particles is limited to~$N$ to make the simulation finite.  (In the
experiment shown, $N=25$.)  The hamiltonian matrix is then diagonal,
with
\begin{equation}
	H_{nn} = \omega_n = n(n-9)/2.
\end{equation}
The eigenvalues range from~$\omega_4=\omega_5=-10$,
through~$\omega_1=\omega_9=0$, up to~$\omega_{25}=200$.

In terms of wave functions and Schr\"odinger's equation, this could
be regarded as the hamiltonian for a single particle, where~$x_n$
was the coefficient of a Lagrange polynomial~$p_n$ (not quite) in
the wave function.  It would be quite a wierd hamiltonian physically,
with terms in products of position and momentum.  On the other hand,
it makes physical sense in second quantisation, where it describes
a collection of particles occupying the same orbital, with mutual
repulsion.

The oscillator was initialised in a quantum-optical coherent state
\cite{glauber}, where the number of particles is Poisson distributed,
with
\begin{equation}\label{eq:vii}
	x_n = e^{-|\alpha|^2/2}\frac{\alpha^n}{\sqrt{n!}}.
\end{equation}
The experiment was run with~$\alpha=2$, which provides some small
initial coefficients with large eigenfrequencies, which are useful
for demonstrating the growth of the parasitic solutions.

\begin{figure}[htbp]
  \centering
  \label{fig:a}\includegraphics[width=\textwidth]{pev.eps}
  \caption{Growth of parasitic eigenvectors.  Dotted line shows the stability limit $\omega=2/h$, dashed line shows the predicted growth rate~$\frac{i}{h}\ln(\frac{h\omega}2)$.  (Tidy up fonts.)}
  \label{fig:i}
\end{figure}

The results of solving this Hamiltonian with the semi-implicit
method, with 4~iterations and time step~$h=0.03$, are shown in
\cref{fig:a}.  The hamiltonian matrix is diagonal, so the norms~$|x_n|$
are constant in the exact solution.  The right hand side of
\cref{eq:vii} is shown as the solid lines at~$t=0$ and~$t=0.09$,
and the values of~$|x_n|$ computed by the semi-implicit formula are
shown as the rulings between these lines.  The size of~$x_n$ is
plotted for all four iterations of the formula within each timestep.
\Cref{eq:viii} shows that, for~$\omega_n\gg 2/h$, the growth of~$x_n$
will still be geometric, resulting in straight lines on the logarithmic
scale of the graph.

Parasitic growth can be
seen at the left hand side, where~$\omega_n$ is large.  The limit
of stability is~$\omega_n=2/h\approx 67$, and the dotted line
indicates~$\omega_17=68$.  The dashed line shows the result of
growing~$|x_n|$ at the predicted rate~$\frac4h\ln(\frac{h\omega}2)$.
The undulations in the lines~$|x_n(t)|$ correspond to the growth
rates given by the different lines in \cref{fig:ii}, which apply
for the fraction of each step ending with a given iteration.

\section{Conclusion}

The semi-implicit formula discussed here is only one of the formulae
that can be used to integrate ordinary differential equations, and
that could be applied to solve \cref{eq:iii} numerically.  However,
most of these formulae suffer from similar problems, and become
unstable when the eigenvalues of the hamiltonian matrix are large
compared to the time step.  The phenomenon of stiffness in discretised
partial differential equations is well known \cite{something}.

There are special formulae to integrate stiff differential equations,
which remain stable when the time steps are long compared to the
eigenvalues of~$H$.  However, these remain stable by ignoring any
physics that is happening on the fast timescales corresponding to
the large eigenvalues.  If the goal is to simulate physics at a
given frequency, then it seems reasonable that the simulation will
require a time step around the corresponding Nyquist rate.

The situation where stiff integration formulae would be appropriate
is when the physics being simulated occurs at slow timescales, but
the method by which the physical state is being represented
inadvertently allows the state to contain a component with a large
energy, that evolves on a fast timescale.  In the numerical example,
this occured due to the coefficient vector~$x$ including coefficients
with large~$n$, for which~$x_n$ was too small to have any effect
on the state.  As demonstrated, the formula caused these minute
components to grow quickly and dominate the actual solution.  It would
be possible to address this by using an integration formula that
was adapted to stiff differential equations.  But it might be better
to address this by changing the representation so that the simulated
state can not include such components, or to modify the hamiltonian
dynamics so that they evolved with reasonable frequencies in the
simulation.

In the example here, the representation was simply an expansion
over energy eigenstates.  This can be modified very simply, by
reducing~$N$ to exclude the large eigenvalues and frequencies.  In
the example, an expansion with coefficients~$x_0$ to~$x_{17}$ would
be stable with the time step used in the example.  However, this
is a contrived example.  If the energy eigenstates were known, their
amplitudes could be calculated directly as complex
exponentials~$e^{i\omega_nt}$, and there would be no need to use a
potentially unstable integration formula.

In real simulations, quantum states are represented by expansion
over a set of states that are not energy eigenstates, by use of a
variational ansatz, or by similar methods.  These representations
are not directly related to the hamiltonian, so it is often the
case that some states with spuriously high energies can be represented,
and the resulting ordinary differential equations are unnecessarily
stiff.  In these simulations, some care must be taken to avoid
inadvertently representing high energy states, or to modify the
hamiltonian dynamics to prevent them from causing stiffness.  Details
for one representation, using Gaussian wave packets, will come in
future papers.

\section*{Acknowledgments}
I would like to acknowledge the support of Swinburne University of Technology.

\bibliographystyle{siamplain}
\bibliography{references}
\end{document}
